\chapter{Deduction of the main results}
\label{chap:tech}

This section contains the deductions of the headline results from the main technical proposition, Propostion~\ref{prop:techmain}.

\begin{theorem}[Solution in sets of positive density]\label{th:density_theorem}
\leanok
\lean{unit_fractions_upper_density}
If $A\subset \bbn$ has positive upper density then there is a finite $S\subset A$ such that $\sum_{n\in S}\frac{1}{n}=1$.
\end{theorem}
\begin{proof}\uses{lem:sieve2,prop:techmain,lem:turan}
  Suppose $A\subset \bbn$ has upper density $\delta>0$. Let $y=C_1/\delta$ and $z=\delta^{-C_2\delta^{-2}}$, where $C_1,C_2$ are two absolute constants to be determined later. It suffices to show that there is some $d\in [y,z]$ and finite $S\subset A$ such that $R(S)=1/d$. Indeed, given such an $S$ we can remove it from $A$ and still have an infinite set of upper density $\delta$, so we can find another $S'\subset A\backslash S$ with $R(S')=1/d'$ for some $d'\in [y,z]$, and so on. After repeating this process at least $\lceil z-y\rceil^2$ times there must be some $d\in [y,z]$ with at least $d$ disjoint $S_1,\ldots,S_d\subset A$ with $R(S_i)=1/d$. Taking $S=S_1\cup\cdots \cup S_d$ yields $R(S)=1$ as required.

  By definition of the upper density, there exist arbitrarily large $N$ such that $\abs{A\cap [1,N]}\geq \frac{\delta}{2}N$. The number of $n\in [1,N]$ divisible by some prime power $q\geq N^{1-6/\log\log N}$ is
  \[\ll N \sum_{N^{1-6/\log\log N}<q\leq N}\frac{1}{q}\ll \frac{N}{\log\log N}\]
  by Mertens' estimate Lemma~\ref{lem:mertens1}. Further, by Tur\'{a}n's estimate Lemma~\ref{lem:turan}
  \[\sum_{n\leq N}(\omega(n)-\log\log N)^2 \ll N \log\log N,\]
  the number of $n\in [1,N]$ that do not satisfy
  \begin{equation}\label{divs}
  \tfrac{99}{100}\log\log N\leq \omega(n) \leq 2\log\log N
  \end{equation}
  is $\ll N/\log\log N$. Finally, provided we choose $C_2$ sufficiently large in the definition of $z$, Lemma~\ref{lem:sieve2} ensures that the proportion of all $n\in \{1,\ldots,N\}$ not divisible by at least two distinct primes $p_1,p_2\in [y,z]$ with $4p_1<p_2$ is at most $\frac{\delta}{8}N$, say.

  In particular, provided $N$ is chosen sufficiently large (depending only on $\delta$), we may assume that $\abs{A_N}\geq \frac{\delta}{4}N$, where $A_N\subset A$ is the set of those $n\in A\cap [N^{1-1/\log\log N},N]$ which satisfy conditions (2)-(4) of Proposition~\ref{prop:techmain}. Since $\abs{A_N}\geq \frac{\delta}{4}N$,
  \[R(A_N) \gg -\log(1-\delta/4)\gg \delta.\]
  In particular, since $y=C_1/\delta$ for some suitably large constant $C_1>0$, we have that $R(A_N)\geq 4/y$, say. All of the conditions of Proposition~\ref{prop:techmain} are now satisfied (provided $N$ is chosen sufficiently large in terms of $\delta$), and hence there is some $S\subset A_N\subset A$ such that $R(S)=1/d$ for some $d\in [y,z]$, which suffices as discussed above.
\end{proof}

\begin{theorem}[Solution in sets of positive logarithmic density, quantitative version]\label{th:log_density_theorem}
\leanok
\lean{unit_fractions_upper_log_density}
There is a constant $C>0$ such that the following holds. If $A\subset \{1,\ldots,N\}$ and
\[\sum_{n\in A}\frac{1}{n}\geq C \frac{\log \log\log N}{\log\log N}\log N\]
then there is an $S\subset A$ such that $\sum_{n\in S}\frac{1}{n}=1$.
\end{theorem}
\begin{proof}
  \uses{lem:sieve1,cor:tech_cor,lem:turan}
  Let $C\geq 2$ be an absolute constant to be chosen shortly, and for brevity let $\epsilon = \log\log\log N/\log\log N$, so that we may assume that $R(A)\geq C\epsilon \log N$. Since $\sum_{n\leq X}\frac{1}{n}\ll \log X$, if $A'=A\cap [N^\epsilon,N]$ we have (assuming $C$ is sufficiently large) $R(A')\geq \frac{C}{2}\epsilon\log N$.

  Let $X$ be those integers $n\in [1,N]$ not divisible by any prime $p\in [5,(\log N)^{1/1200}]$. Lemma~\ref{lem:sieve1} implies that, for any $x\geq \exp(\sqrt{\log N})$,
  \[\abs{X\cap[x,2x)}\ll \frac{x}{\log\log N}\]
  and hence, by partial summation,
  \[\sum_{\substack{n\in X\\ n\in [\exp(\sqrt{\log N}),N]}}\frac{1}{n}\ll \frac{\log N}{\log\log N}.\]
  Similarly, if $Y$ is the set of those $N\in [1,N]$ such that $\omega(n) <\frac{99}{100}\log\log N$ or $\omega(n)\geq \frac{101}{100}\log\log N$ then Tur\'{a}n's estimate Lemma~\ref{lem:turan}
  \[\sum_{n\leq x}(\omega(n)-\log\log n)^2\ll x\log\log x\]
  implies that $\abs{Y\cap [x,2x)}\ll x/\log\log N$ for any $N\geq x\geq \exp(\sqrt{\log N})$, and so
  \[\sum_{\substack{n\in Y\\ n\in [\exp(\sqrt{\log N}),N]}}\frac{1}{n}\ll \frac{\log N}{\log\log N}.\]
  In particular, provided we take $C$ sufficiently large, we can assume that $R(A'\backslash (X\cup Y))\geq \frac{C}{4}\epsilon \log N$, say.

  Let $\delta=1-1/\log\log N$, and let $N_i=N^{\delta^i}$, and $A_i=(A'\backslash (X\cup Y))\cap [N_{i+1},N_i]$. Since $N_i\leq N^{e^{-i/\log\log N}}$ and $A'$ is supported on $n\geq N^\epsilon$, the set $A_i$ is empty for $i> \log(1/\epsilon)\log\log N$, and hence by the pigeonhole principle there is some $i$ such that
  \[R(A_i)\geq \frac{C}{8}\frac{\epsilon\log N}{(\log\log N)\log(1/\epsilon)}.\]
  By construction, $A_i\subset [N_{i+1},N_i]\subset [N_i^{1-1/\log\log N_i},N_i]$, and every $n\in A_i$ is divisible by some prime $p$ with $5\leq p\leq (\log N)^{1/1200}\leq (\log N_i)^{1/500}$. Furthermore, every $n\in A_i$ satisfies $\omega(n)\geq \frac{99}{100}\log\log N\geq \frac{99}{100}\log\log N_i$ and $\omega(n)\leq \frac{101}{99}\log\log N\leq 2\log\log N_i$.

  Finally, it remains to discard the contribution of those $n\in A_i$ divisible by some large prime power $q> N_i^{1-6/\log\log N_i}$. The contribution to $R(A_i)$ of all such $n$ is at most
\[
  \sum_{N_i^{1-6/\log\log N_i}< q\leq N_i}\sum_{\substack{n\leq N_i\\ q\mid n}}\frac{1}{n}
  \ll \sum_{N_i^{1-6/\log\log N_i}< q\leq N_i}\frac{\log(N_i/q)}{q}\]
  \[\ll \frac{\log N_i}{\log\log N_i}\sum_{N_i^{1-6/\log\log N_i}<q\leq N_i}\frac{1}{q}\ll \frac{\log N}{(\log\log N)^2},\]
  using Lemma~\ref{lem:mertens1}. Provided we choose $C$ sufficiently large, this is $\leq R(A_i)/2$, and hence, if $A_i'\subset A_i$ is the set of those $n$ divisible only by prime powers $q\leq N_i^{1-6/\log\log N_i}$, then $R(A_i')\geq (\log N)^{1/200}$, say. All of the conditions of Corollary~\ref{cor:tech_cor} are now met, and hence there is some $S\subset A_i'\subset A$ such that $R(S)=1$, as required.
\end{proof}


\begin{corollary}[Useful Technical Corollary]\label{cor:tech_cor}
  \lean{corollary_one}
  \leanok
  Suppose $N$ is sufficiently large and $A\subset [N^{1-1/\log\log N},N]$ is such that
  \begin{enumerate}
  \item $R(A)\geq 2(\log N)^{1/500}$,
  \item every $n\in A$ is divisible by some prime $p$ satisfying $5 \leq p \leq (\log N)^{1/500}$,
  \item every prime power $q$ dividing some $n\in A$ satisfies $q\leq N^{1-6/\log\log N}$, and
  \item every $n\in A$ satisfies
  \[\tfrac{99}{100}\log\log N\leq \omega(n) \leq 2\log\log N.\]
  \end{enumerate}
  There is some $S\subset A$ such that $R(S)=1$.
\end{corollary}
\begin{proof}
  \uses{prop:techmain}
  \leanok
  Let $k$ be maximal such that there are disjoint $S_1,\ldots,S_k\subset A$ where, for each $1\leq i\leq k$, there exists some $d_i\in [1,(\log N)^{1/500}]$ such that $R(S_i)=1/d_i$. Let $t(d)$ be the number of $S_i$ such that $d_i=d$. If there is any $d$ with $t(d)\geq d$ then we are done, taking $S$ to be the union of any $d$ disjoint $S_j$ with $R(S_j)=1/d$. Otherwise,
  \[\sum_i R(S_i)= \sum_{1\leq d\leq (\log N)^{1/500}} \frac{t(d)}{d}\leq (\log N)^{1/500},\]
  and hence if $A'=A\backslash (S_1\cup\cdots \cup S_k)$ then $R(A')\geq (\log N)^{1/500}$.

  We may now apply Proposition~\ref{prop:techmain} with $y=1$ and $z=(\log N)^{1/500}$ -- note that condition (2) of Proposition~\ref{prop:techmain} follows from condition (2) of the hypotheses with $d_1=1$ and $d_2=p\in [5,(\log N)^{1/500}]$ some suitable prime divisor. Thus there exists some $S'\subset A'$ such that $R(S')=1/d$ for some $d\in [1,(\log N)^{1/500}]$, contradicting the maximality of $k$.
\end{proof}

\section{Sieve Lemmas}

\begin{lemma}[Sieve Estimate 1]\label{lem:sieve1}
\leanok
\lean{sieve_lemma_one}
Let $N$ be sufficiently large and $z,y$ be two parameters such that $\log N \geq z>y\geq 3$. If $X$ is the set of all those integers not divisible by any prime in $p\in [y,z]$ then
\[\abs{ X\cap [N,2N)}\ll \frac{\log y}{\log z}N.\]
\end{lemma}
\begin{proof}\uses{lem:sieve_eratosthenes,lem:mertens2}
  Lemma~\ref{lem:sieve_eratosthenes} yields
  \[\abs{X\cap [N,2N)} = \prod_{y\leq p\leq z}\brac{1-\frac{1}{p}}N+ O(2^z).\]
  Mertens' estimate \ref{lem:mertens2} yields
  \[\prod_{p\leq z}\brac{1-\frac{1}{p}}^{-1} \gg \log z\]
  and
  \[\prod_{p\leq y}\brac{1-\frac{1}{p}}^{-1} \ll \log y,\]
  whence
  \[\prod_{y\leq p\leq z}\brac{1-\frac{1}{p}}^{-1}\gg \frac{\log z}{\log y},\]
  and hence
  \[ \prod_{y\leq p\leq z}\brac{1-\frac{1}{p}}\ll \frac{\log y}{\log z}.\]
  Therefore the first term above is $\ll \frac{\log y}{\log z}N$. The second term is
  \[\ll 2^z \leq 2^{\log N}=N^{\log 2}\ll \frac{N}{\log N}\ll \frac{\log y}{\log z}N,\]
  and the result follows.
\end{proof}


\begin{lemma}[Sieve Estimate 2]\label{lem:sieve2}
  \leanok
  \lean{sieve_lemma_two}
Let $N$ be sufficiently large and $z,y$ be two parameters such that $(\log N)^{1/2}\geq z>4y\geq 8$. If $Y\subset [1,N]$ is the set of all those integers divisible by at least two distinct primes $p_1,p_2\in [y,z]$ where $4p_1<p_2$ then
\[\abs{ \{1,\ldots,N\}\backslash Y}\ll \brac{\frac{\log y}{\log z}}^{1/2}N.\]
\end{lemma}
\begin{proof}\uses{lem:sieve1}
  Let $w\in (4y,z)$ be some parameter to be chosen later. Lemma~\ref{lem:sieve1} implies that the number of $n\in \{1,\ldots,N\}$ not divisible by any prime $p\in [w,z]$ is $\ll \frac{\log w}{\log z}N$.

  Similarly, for any $p\in [w,z]$, the number of those $n\in [1,N]$ divisible by $p$ and no prime  $q\in [y,p/4)$ is
  \[\ll \frac{\log y}{\log p}\frac{N}{p}.\]
  It follows that the number of $n\in\{1,\ldots,N\}\backslash Y$ is
  \[\ll \brac{\frac{\log w}{\log z}+ \log y\sum_{p\geq w}\frac{1}{p\log p}}N.\]
  By partial summation, $\sum_{p\geq w}\frac{1}{p\log p}\ll 1/\log w$, and hence
  \[\abs{ \{1,\ldots,N\}\backslash Y}\ll \brac{\frac{\log w}{\log z}+\frac{ \log y}{\log w}}N.\]
  Choosing $w=\exp\brac{\sqrt{(\log y)(\log z)}}$ completes the proof.
\end{proof}


