\chapter{Definitions}
\label{chap:def}

Some definitions, local to this paper, that occur frequently.

\begin{definition}
  \label{def:rec_sum}
  \lean{rec_sum}
  \leanok
  For any finite $A\subset \bbn$
  \[R(A)=\sum_{n\in A}\frac{1}{n}.\]
\end{definition}

\begin{definition}
  \label{def:local_part}
  \lean{local_part}
  \leanok
  For any finite $A\subset \bbn$ and prime power $q$ we define
  \[A_q = \{ n\in A : q\mid n\textrm{ and }(q,n/q)=1\}\]
  and let $\mathcal{Q}_A$ be the set of all prime powers $q$ such that $A_q$ is non-empty (i.e. those $p^r$ such that $p^r\| n$ for some $n\in A$).
\end{definition}

\begin{definition}
  \label{def:rec_sum_local}
  \lean{rec_sum_local}
  \leanok
  \uses{def:local_part}
  For any finite $A\subset \bbn$ and prime power $q\in\mathcal{Q}_A$ we define
  \[R(A;q) = \sum_{n\in A_q}\frac{q}{n}.\]
\end{definition}

\begin{definition}
  \label{def:interval_rare_ppowers}
  \lean{interval_rare_ppowers}
  \leanok
  \uses{def:local_part}
  For any finite set $A\subset \bbn$, $K\in \mathbb{R}$ and interval $I$, we define $\mathcal{D}_I(A;K)$ to be the set of those $q\in\mathcal{Q}_A$ such that
  \[\#\{ n\in A_q: \textrm{no element of }I\textrm{ is divisible by }n\}\ <\ K / q.\]
\end{definition}
