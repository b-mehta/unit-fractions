\chapter{Fourier Analysis}
\label{chap:fourier}

This section contains the part of the proof that uses Fourier analysis to reduce finding solutions to a combinatorial problem involving divisors.

In this section, we write $[A]$ for the lowest common multiple of $A$ (if $A$ is a finite set of naturals).
\section{Local definitions}

\begin{definition}
  \label{def:integer_count}
  \uses{def:rec_sum}
  \leanok
  \lean{integer_count}
For any finite set $A$ of natural numbers and integer $k\geq 1$ we write $F(A;k)$ for the count of the number of subsets $S\subset A$ such that $kR(S)$ is an integer.
\end{definition}

\begin{definition}
  \label{def:j}
  \leanok
  \lean{j}
  Let $J(A) = (-[A]/2,[A]/2]\cap\bbz \backslash \{0\}$.
\end{definition}

\begin{definition}
  \label{def:cos_prod}
  \leanok
  \lean{cos_prod}
  For any finite set of naturals $B$ and integer $t$, we define $C(B;t) = \prod_{n\in B}\abs{\cos(\pi t/n)}$.
\end{definition}

\begin{definition}
  \label{def:major_arc}
  \leanok
  \lean{major_arc}
  \uses{def:j}
  For any finite set of naturals $A$ and integer $k\geq 1$, and real $K>0$, we define the `major arc' corresponding to $t\in\bbz$ as
\[\mathfrak{M}(t;A,k,K) = \{ h\in  J(A): \abs{h-t[A]/k}\leq K/2k\}\]
  Let $\mathfrak{M}(A,k,K) = \cup_{t\in \bbz}\mathfrak{M}(t;A,k,K)$.
\end{definition}

\begin{definition}
  \label{def:minor_arcs}
  \uses{def:major_arc}
  \leanok
  \lean{minor_arc₁}
Let $I_h(K,k)$ be the interval of length $K$ centred at $kh$, and let $\mathfrak{m}_1(A,k,K,\delta)$ be those $h\in J(A)\backslash \mathfrak{M}(A,k,K)$ such that
\[\# \{ n\in A : \textrm{no element of }I_h(K,k)\textrm{ is divisible by }n\}\geq \delta.\]
Let $\mathfrak{m}_2(A,k,K,\delta)$ be the rest of $J(A)\backslash\mathfrak{M}(A,k,K)$.
\end{definition}

\section{Precursor general lemmas}

\begin{lemma}\label{lem:orthogonality}
  \leanok
  \lean{orthogonality}
For any $n,m\in\bbn$, if $e(x) = e^{2\pi ix}$, then if $I$ is any set of $m$ consecutive integers then
\[1_{m\mid n}=\frac{1}{m}\sum_{h\in I}e(h n/m).\]
\end{lemma}
\begin{proof}
  \leanok
If $m\mid n$ then $n/m$ is an integer, so $e(hn/m)=1$ for all $h\in \mathbb{Z}$, so the right-hand side is $1$.

If $m\nmid n$ then $e(n/m)\neq 1$. Let $S=\sum_{h\in I}e(hn/m)$, so it suffices to show that $S=0$. Using $e(x+y)=e(x)e(y)$ we have
\[e(n/m)S = \sum_{h\in I}e((h+1)n/m).\]
If $r=\min(I)$ and $s=\max(I)$ then the right-hand side is $S + e((s+1)n/m)-e(rn/m)$. But since $I$ is a set of $m$ consecutive integers we know that $s=r+m-1$, and so
\[e((s+1)n/m)-e(rn/m) = e(rn/m+n)-e(rn/m)=e(rn/m)e(n)-e(rn/m)=0,\]
since $e(n)=1$. Therefore $e(n/m)S=S$, and hence since $e(n/m)\neq 1$, this forces $S=0$ as required.
\end{proof}

\begin{lemma}\label{lem:lcm_desc}
  \lean{lcm_desc}
  \leanok
Let $A$ be a finite set of natural numbers not containing $0$. If
\[\mathcal{P}_A = \{ p \textrm{ prime}: \exists n\in A : p \mid n\}\]
and for all $p\in \mathcal{P}_A$ then $r_p\geq 0$ is the greatest integer such that $p^{r_p}$ divides some $n\in A$, then
\[[A]=\prod_{p\in\mathcal{P}_A}p^{r_p}.\]
\end{lemma}
(NOTE: This is a generally useful fact when working with lowest common multiples, perhaps this should be in mathlib somewhere.)
\begin{proof}
  \leanok
We first note that if $p\mid [A]$ then $p\in \mathcal{P}_A$. If not, suppose $p\not\in\mathcal{P}_A$ and $p\mid [A]$. Let $M=[A]/p$. We claim every $n\in A$ divides $M$, contradicting the definition of lowest common multiple. It suffices to show that if a prime power $q^r\mid n$ then $q\mid M$. But we know $q^r\mid [A]=Mp$, and $(q^r,p)=1$, so $q\mid M$ as required.

By the fundamental theorem of arithmetic, we can write $[A] = \prod_{p\in \mathcal{P}_A}p^{s_p}$ for some integers $s_p\geq 0$. It remains to show that $s_p=r_p$.

That $s_p\geq r_p$ follows from the fact that $p^{r_p}\mid n$ for some $n\in A$ by definition, hence $p^{r_p}\mid [A]$, hence $r_p\leq s_p$.

Suppose that $s_p>r_p$, and as above consider $M=[A]/p$. We claim every $n\in A$ divides $M$, the required contradiction. If a prime power $q^r\mid n$ with $q\neq p$ then $q\mid Mp$ hence $q\mid M$. If $p^r\mid n$ then $r\leq r_p$, so $p^r\mid p^{r_p}\mid p^{s_p-1}\mid M$, as required.
\end{proof}

\begin{lemma}\label{lem:smooth_lcm}
  \uses{def:local_part}
  \lean{smooth_lcm}
  \leanok
If $A$ is a finite set of natural numbers not containing $0$ such that if $q\in\mathcal{Q}_A$ then $q\leq X$, then $[A]\leq e^{O(X)}$.
\end{lemma}
\begin{proof}
  \leanok
  \uses{lem:chebyshev,lem:lcm_desc}
  We have $[A]=\prod_{p\in\mathcal{P}_A}p^{r_p}$, where $p^{r_p}\in \mathcal{Q}_A$, by Lemma~\ref{lem:lcm_desc}. By hypothesis $p^{r_p}\leq X$ and so $[A]\leq X^{\abs{\mathcal{P}_A}}$. The set $\mathcal{P}_A$ is a subset of all primes $\leq X$, and so $\abs{\mathcal{P}_A}\leq \pi(X)\ll X/\log X$ by Chebyshev's estimate Lemma~\ref{lem:chebyshev}. Therefore
  \[[A]\leq X^{O(X/\log X)}=e^{O(X)}.\]
\end{proof}

\begin{lemma}\label{lem:cos_bound}
  \leanok
  \lean{cos_bound}
If $x\in [0,1/2]$ then
\[\cos(\pi x) \leq e^{-2x^2}.\]
\end{lemma}
\begin{proof}
  \leanok
We have Jordan's inequality, which says $\sin(\pi x)\geq 2x$ for all $x\in[0,1/2]$. Therefore
\[\cos(\pi x)^2 = 1-\sin(\pi x)^2\leq 1-4x^2.\]
Since $1-y\leq e^{-y}$ for all $y\geq 0$, the right-hand side is $\leq e^{-4x^2}$. Taking square roots, and using $\cos(\pi x)\geq 0$ for all $x\in[0,1/2]$, yields
\[\cos(\pi x) \leq e^{-2x^2}.\]
\end{proof}

\begin{lemma}\label{lem:triv_q_bound}
  \lean{triv_q_bound}
  \leanok
  \uses{def:local_part}
Let $A$ be a finite set of naturals not containing $0$. For any $n\in A$, let $\mathcal{Q}_A(n)$ denote all those $q\in \mathcal{Q}_A$ such that $n\in A_q$, then
\[\abs{\mathcal{Q}_A(n)}\leq  \tfrac{1}{\log 2}\log n.\]
\end{lemma}
\begin{proof}
  \leanok
The first step is to show that $\prod_{q\in\mathcal{Q}(n)}q\mid n$ (in fact they're equal, but all we need is the one direction). This can be shown by showing that every $q\in\mathcal{Q}(n)$ divides $n$, since $n\in A_q$ implies $q\mid n$,  and then noting that any two distinct $q_1,q_2\in \mathcal{Q}(n)$ are coprime. If not, then since they are prime powers, they must both be powers of the same prime, say $q_1=p^{r_1}< q_2=p^{r_2}$. Since $n\in A_{q_2}$ we have $p^{r_2}\mid n$, but then $p\mid n/q_1$, so $p\mid (q_1,n/q_1)$, and so $n\not\in A_{q_1}$.

Therefore $\prod_{q\in\mathcal{Q}(n)}q\leq n$. Since all prime powers are $\geq 2$ it follows that $2^{\abs{\mathcal{Q}(n)}}\leq n$, and the result follows taking logarithms.
\end{proof}

\section{Towards the main proposition}

\begin{lemma}\label{lem:orthog_rat}
  \uses{def:integer_count}
  \leanok
  \lean{orthog_rat}
If $A$ is a finite set of natural numbers not containing $0$ and $k\geq 1$ is an integer then
\[F(A;k)= \frac{1}{[A]}\sum_{-[A]/2< h\leq [A]/2}\prod_{n\in A}(1+e(kh/n)).\]
\end{lemma}
\begin{proof}
  \leanok
\uses{lem:orthogonality}
For any $S\subset A$,  $k\sum_{n\in S}\frac{1}{n}\in \bbz$ if and only if $k\sum_{n\in S}\frac{[A]}{n}\in [A]\cdot \bbz$. By definition $n\mid [A]$ for all $n\in A$, and so $k\sum_{n\in S}\frac{[A]}{n}$ is an integer. It is $\geq 0$ since each summand is. Therefore by Lemma~\ref{lem:orthogonality}, if $I$ is any set of $[A]$ consecutive integers
\[1_{k\sum_{n\in S}\frac{1}{n}\in \bbz}=1_{[A]\mid k\sum_{n\in S}\frac{[A]}{n}} = \frac{1}{[A]}\sum_{h\in I}e(kh\sum_{n\in S}\frac{1}{n}).\]
Therefore, changing summation,
\[F(A;k) = \sum_{S\subset A}1_{k\sum_{n\in S}\frac{1}{n}\in \bbz} = \frac{1}{[A]} \sum_{h\in I}\sum_{S\subset A}\prod_{n\in S}e(kh/n).\]
The lemma now follows choosing $I=(-[A]/2,[A]/2]\cap\bbz$, and using the general fact that for any indexed set of complex numbers $(x_i)_{i\in I}$
\[\sum_{J\subset I}\prod_{j\in J}x_j = \prod_{i\in I}(1+x_i).\]
\end{proof}

\begin{lemma}\label{lem:orthog_simp}
  \leanok
  \lean{orthog_simp}
If $k\geq 1$ is an integer and $A$ is a finite set of natural numbers such that there is no $S\subset A$ such that $R(S)=1/k$, and $R(A)<2/k$, then
\[\sum_{-[A]/2< h\leq [A]/2}\Re\brac{\prod_{n\in A}(1+e(kh/n))}= [A].\]
\end{lemma}
\begin{proof}
  \leanok
  \lean{orthog_simp}
\uses{lem:orthog_rat}
For any $S\subset A$ we have $k\sum_{n\in S}\frac{1}{n}\leq kR(A)<2$, and therefore if $k\sum_{n\in S}\frac{1}{n} \in \bbn$ then $k\sum_{n\in S}\frac{1}{n} =0$ or $=1$. The latter can't happen by assumption, so $k\sum_{n\in S}\frac{1}{n} =0$. A non-empty sum of $>0$ summands is $>0$, so if $k\sum_{n\in S}\frac{1}{n}\in \bbz$ then $S=\emptyset$. Therefore $F(A;k)=1$.

By Lemma~\ref{lem:orthog_rat} therefore
\[1=\frac{1}{[A]}\sum_{-[A]/2< h\leq [A]/2}\prod_{n\in A}(1+e(kh/n)).\]
The conclusion follows multiplying both sides by $[A]$ and taking real parts of both sides.
\end{proof}

\begin{lemma}\label{lem:orthog_simp2}
  \uses{def:j}
  \leanok
  \lean{orthog_simp2}
If $k\geq 1$ is an integer and $A$ is a finite set of natural numbers such that there is no $S\subset A$ such that $R(S)=1/k$, and $R(A)<2/k$, and $[A]\leq 2^{\abs{A}-1}$ then
\[\sum_{h\in J(A)}\Re\brac{\prod_{n\in A}(1+e(kh/n))}\leq -2^{\abs{A}-1}.\]
\end{lemma}
\begin{proof}
  \leanok
  \lean{orthog_simp2}
\uses{lem:orthog_simp}
By Lemma~\ref{lem:orthog_simp}
\[\sum_{-[A]/2< h\leq [A]/2}\Re\brac{\prod_{n\in A}(1+e(kh/n))}= [A].\]
By assumption the right-hand side is $\leq 2^{\abs{A}-1}$.

When $h=0$
\[\Re\brac{\prod_{n\in A}(1+e(kh/n))}= \Re\brac{\prod_{n\in A}2}=2^{\abs{A}}.\]
Therefore
\[\sum_{h\in J(A)}\Re\brac{\prod_{n\in A}(1+e(kh/n))}+2^{\abs{A}}\leq 2^{\abs{A}-1},\]
and the result follows after rearranging.
\end{proof}

\begin{lemma}\label{lem:majorarcs_disjoint}
  \uses{def:major_arc}
  \leanok
  \lean{majorarcs_disjoint}
If $A$ is a finite set of integers and $K$ is a real such that $[A]>K$ then, for any integer $k\geq 1$, the sets $\mathfrak{M}(t;A,k,K)$ are disjoint for distinct $t\in \bbz$.
\end{lemma}
\begin{proof}
  \leanok
Suppose not and $h\in \mathfrak{M}(t_1)\cap \mathfrak{M}(t_2)$. By definition,
\[\abs{hk-t_1[A]}\leq K/2\]
and
\[\abs{hk-t_2[A]}\leq K/2,\]
and so by the triangle inequality $[A]\abs{t_1-t_2}\leq K$. Since $t_1\neq t_2$, we know $\abs{t_1-t_2}\geq 1$, and so $[A]\leq K$, contradicting the assumption.
\end{proof}

\begin{lemma}\label{lem:useful_rewrite}
  \leanok
  \lean{useful_rewrite}
  \uses{def:rec_sum}
For any finite set of natural numbers $A$ and $\theta\in \mathbb{R}$
\[\Re\brac{\prod_{n\in A}(1+e(\theta/n))}=2^{\abs{A}}\cos(\pi \theta R(A))\prod_{n\in A}\cos(\pi \theta/n).\]
\end{lemma}
\begin{proof}
  \leanok
Rewrite each factor in the product using $1+e(\theta/n)=2e(\theta/2n)\cos(\pi \theta/n)$, so
\[\Re\brac{\prod_{n\in A}(1+e(\theta/n))} = \Re\brac{ 2^{\abs{A}}e(\theta R(A)/2)\prod_{n\in A}\cos(\pi \theta/n)}.\]
Taking out real factors, this is $2^{\abs{A}}\prod_{n\in A}\cos(\pi \theta/n)\Re e(\theta R(A)/2)$, and the claim follows.
\end{proof}

\begin{lemma}\label{lem:majorarcs}
  \leanok
  \lean{majorarcs}
Let $M\geq 1$ and $A$ a finite set of naturals such that $n\geq M$ for all $n\in A$. Let $K$ be a real such that $K<M$. Let $k\geq 1$ be an integer which divides $[A]$. Suppose that $kR(A) \in [2-k/M,2)$.
\[\sum_{h\in \mathfrak{M}(A,k,K)}\Re\brac{\prod_{n\in A}(1+e(kh/n))} \geq 0.\]
\end{lemma}
\begin{proof}
  \leanok
\uses{lem:majorarcs_disjoint,lem:useful_rewrite}
Since $k$ divides $[A]$, we know that $t[A]/k$ is an integer for any $t\in \bbz$, and hence by definition of $\mathfrak{M}(t)$ we can write $h=t[A]/k+r$, where $r$ is an integer satisfying $\abs{r}\leq K/2k$. Therefore, letting
\[J_t =[-K/2k,K/2k]\cap (J(A)-t[A]/k),\]
then
\[\sum_{h\in \mathfrak{M}(t)}\Re\brac{\prod_{n\in A}(1+e(kh/n))}=\sum_{r\in J_t}\Re\brac{\prod_{n\in A}(1+e((t[A]+rk)/n))}=\sum_{r\in J_t}\Re\brac{\prod_{n\in A}(1+e(rk/n))},\]
since $t[A]/n$ is always an integer, by definition of $[A]$.

Using Lemma~\ref{lem:useful_rewrite}, this is
\[2^{[A]}\sum_{h\in \mathfrak{M}(t)}\cos(\pi krR(A))\prod_{n\in A}\cos(\pi kr/n).\]


Since $[A]\geq \min(A)\geq M>K$, the hypotheses of Lemma~\ref{lem:majorarcs_disjoint} are all met, and so $\mathfrak{M}$ is the disjoint union of $\mathfrak{M}(t)$ as $t$ ranges over $t\in \bbz$. Therefore $1_{h\in\mathfrak{M}}=\sum_t 1_{h\in \mathfrak{M}(t)}$, and therefore using the above and rearranging the sum,
\[\sum_{h\in \mathfrak{M}(A,k,K)}\Re\brac{\prod_{n\in A}(1+e(kh/n))}
=\sum_{r\in [-K/2k,K/2k]\cap\bbz}\brac{\sum_{t\in \bbz}1_{r\in J_t}}\cos(\pi krR(A))\prod_{n\in A}\cos(\pi kr/n).\]
Since for all $n\in A$ we have $n\geq M>K$ we have $\abs{kr/n}< 1/2$ for all $r$ with $\abs{r}\leq K/2k$, and hence $\cos(\pi kr/n)\geq 0$ for all such $n$ and $r$.

Furthermore, writing $kR(A)=2-\epsilon$ for some $0<\epsilon\leq k/M$, we have (since $r$ is an integer)
\[\cos(\pi kr R(A)) = \cos(-\pi r\epsilon)\geq 0,\]
since $\abs{r\epsilon}\leq K/2M<1/2$ for all $\abs{r}\leq K/2k$. It follows that
\[\brac{\sum_{t\in \bbz}1_{r\in J_t}}\cos(\pi krR(A))\prod_{n\in A}\cos(\pi kr/n)\geq 0\]
for all $r\in [-K/2k,K/2k]\cap \bbz$, and hence as the sum of non-negative summands the original sum is non-negative as required.
\end{proof}

\begin{lemma}\label{lem:minor_lbound}
  \leanok
  \lean{minor_lbound}
  \uses{def:cos_prod}
Let $M\geq 1$ and $A$ a finite set of naturals such that $n\geq M$ for all $n\in A$. Let $K$ be a real such that $K<M$. Let $k\geq 1$ be an integer which divides $[A]$. Suppose that $kR(A) \in [2-k/M,2)$, and there is no $S\subset A$ such that $R(S)=1/k$, and $[A]\leq 2^{\abs{A}-1}$. Then
\[\sum_{h\in J(A)\backslash \mathfrak{M}(A,k,K)} C(A;hk)\geq 1/2.\]
\end{lemma}
\begin{proof}
  \leanok
\uses{lem:majorarcs,lem:orthog_simp2}
By Lemma~\ref{lem:orthog_simp2}
\[\sum_{h\in J(A)}\Re\brac{\prod_{n\in A}(1+e(kh/n))}\leq -2^{\abs{A}-1}.\]
By Lemma~\ref{lem:majorarcs}
\[\sum_{h\in \mathfrak{M}(A,k,K)}\Re\brac{\prod_{n\in A}(1+e(kh/n))} \geq 0.\]
Therefore
\[\sum_{h\in J(A)\backslash \mathfrak{M}(A,k,K)}\Re\brac{\prod_{n\in A}(1+e(kh/n))}\leq -2^{\abs{A}-1}.\]
By the triangle inequality, using $\abs{\Re z}\leq \abs{z}$ and $\abs{1+e(\theta)}=2\cos(\pi \theta)$,
\[\sum_{h\in J(A)\backslash \mathfrak{M}(A,k,K)}\abs{\cos(\pi kh/n)} \geq 1/2\]
as required.
\end{proof}

\begin{lemma}\label{lem:cos_prod_bound}
  \leanok
  \lean{cos_prod_bound}
  \uses{def:cos_prod,def:interval_rare_ppowers}
For any finite set $A$ such that $n\leq N$ for all $n\in A$, and integer $t$, if $t\equiv t_n\pmod{n}$ for $\abs{t_n}\leq n/2$ for all $n\in A$, then
\[C(A;t)\leq \exp\brac{-\frac{2}{N^2}\sum_{n\in A}t_n^2}.\]
\end{lemma}
\begin{proof}
  \leanok
\uses{lem:cos_bound}
We first note that $\abs{\cos(\pi t/n)}=\abs{\cos(\pi t_n/n)}$ for all $n\in A$, by periodicity of cosine. By Lemma~\ref{lem:cos_bound}, therefore
\[\abs{\cos(\pi t/n)}\leq \exp(-2t_n^2/n^2)\leq \exp(-\tfrac{2}{N^2}t_n^2),\]
and the lemma follows taking the product over all $n\in A$.
\end{proof}


\begin{lemma}\label{lem:minor1_bound}
  \uses{def:minor_arcs}
  \leanok
  \lean{minor1_bound}
Suppose that $N$ is sufficiently large and $M\geq 8$. Let $T>0$ be a real and $k\geq 1$ be an integer. Let $A\subset [M,N]$ be a set of integers such that if $q\in\mathcal{Q}_A$ then $q\leq \frac{TK^2}{N^2 \log N}$. Then
\[\sum_{h\in \mathfrak{m}_1(A,k,K,T)} C(A;hk)\leq 1/8.\]
\end{lemma}
\begin{proof}
  \leanok
\uses{lem:smooth_lcm,lem:cos_prod_bound}
We show in fact that for any $h\in \mathfrak{m}_1(A,k,K,T)$ we have
\[C(A;hk)\leq 1/[A]^2.\]
The result then immediately follows since $\abs{\mathfrak{m}_1(A,k,K,T)}\leq \abs{J(A)}\leq [A]$, assuming $[A]\geq 8$, which is true since $[A]\geq \min(A)\geq M\geq 8$.

By Lemma~\ref{lem:cos_prod_bound},
\[C(A;hk)\leq \exp\brac{-\frac{2}{N^2}\sum_{n\in A}h_n^2},\]
where $kh\equiv h_n\pmod{n}$ and $\abs{h_n}\leq n/2$. Let $I_h$ be the interval of length $K$ centred around $kh$. If no element of $I_h$ is divisible by $n$ then $\abs{h_n}>K/2$. Therefore, by definition of $\mathfrak{m}_1$, $\abs{h_n}>K/2$ for at least $T$ many $n\in A$, and hence $\sum_{n\in A}h_n^2\geq K^2T/4$, and so
\[C(A;h)\leq \exp\brac{-\frac{K^2T}{2N^2}}.\]
It remains to note that by Lemma~\ref{lem:smooth_lcm}
\[[A]\leq \exp\brac{O\brac{\frac{K^2T}{N^2\log N}}}\leq \exp\brac{\frac{K^2T}{4N^2}}\]
assuming $N$ is sufficiently large.
\end{proof}

\begin{lemma}\label{lem:minor2_ind_bound}
  \leanok
  \lean{minor2_ind_bound}
Suppose that $N\geq 4$. Let $A\subset [1,N]$ be a finite set of integers and $t$ an integer. Let $K,L>0$ be reals and suppose that $q\leq \tfrac{1}{16}LK^2/N^2(\log N)^2$ for all $q\in \mathcal{Q}_A$. Let $\mathcal{D}=\mathcal{D}_I(A;L)$ where $I$ is the interval of length $K$ centred at $t$. Then
\[C(A;t)\leq N^{-4\abs{\mathcal{Q}_A\backslash\mathcal{D}}}.\]
\end{lemma}
\begin{proof}
  \leanok
\uses{lem:cos_prod_bound,lem:triv_q_bound}
For any $n\in A$, let $\mathcal{Q}(n)$ denote all those $q\in \mathcal{Q}$ such that $n\in A_q$. Therefore, for any real $x_n\geq 0$,
\[\prod_{n\in A}x_n=\prod_{n\in A}\prod_{q\in \mathcal{Q}(n)}x_n^{1/\abs{\mathcal{Q}(n)}}
=\prod_{q\in\mathcal{Q}}\prod_{n\in A_q}x_n^{1/\abs{\mathcal{Q}(n)}}.\]
By Lemma~\ref{lem:triv_q_bound} for any $n\in A$ we have $\abs{\mathcal{Q}(n)}\leq \tfrac{1}{\log 2}\log n\leq 2\log N$, and so if $0\leq x_n\leq 1$,
\[\prod_{n\in A}x_n\leq \prod_{q\in\mathcal{Q}}\prod_{n\in A_q}x_n^{1/2\log N}=\prod_{q\in \mathcal{Q}}\brac{\prod_{n\in A_q}x_n}^{1/2\log N}.\]
In particular,
 \[C(A;t)\leq \prod_{q\in \mathcal{Q}}C(A_q;t)^{/2\log N}.\]
 Using the trivial bound $C(A_q;t)\leq 1$, to prove the lemma it therefore suffices to show that for every $q\in\mathcal{Q}\backslash\mathcal{D}_h$ we have $C(A_q;t)\leq N^{-8\log N}$.

For any $n\in A$ let $t\equiv t_n\pmod{n}$, where $\abs{t_n}\leq n/2$. For any $q\in\mathcal{Q}\backslash \mathcal{D}$ there are, by definition of $\mathcal{D}$, at least $L/q$ many $n\in A_q$ such that $n$ divides no element of the interval of length $K$ centred at $t$.

Recall that $t_n$ is the integer in $(-n/2,n/2]$ such that $t\equiv t_n\pmod{n}$, so that $n$ divides $t-t_n$. If $\abs{t_n}\leq K/2$ then $t-t_n$ is in the interval of length $K$ centred at $t$, which contradicts the previous paragraph. Therefore $\abs{t_n}>K/2$. Hence by Lemma~\ref{lem:cos_prod_bound}
\[C(A_q;t)\leq\exp\brac{-\frac{2}{N^2}\cdot \frac{L}{q}\cdot \frac{K^2}{4}}.\]
By assumption, $q\leq LK^2/16N^2(\log N)^2$, and the proof is complete.
\end{proof}

\begin{lemma}\label{lem:minor2_bound}
Let $N$ be a sufficiently large integer. Let $K,L>0$ be reals such that $0<K\leq N$. Let $T>0$ be any real. Let $k$ be an integer such that $1\leq k \leq N/64$.  Let $A\subset [1,N]$ be a finite set of integers such that
\begin{enumerate}
\item if $q\in\mathcal{Q}_A$ then $q\leq \tfrac{1}{16}\frac{LK^2}{N^2(\log N)^2}$,  and
\item for any interval $I$ of length $K$, either
\begin{enumerate}
\item \[\# \{ n\in A : \textrm{no element of }I\textrm{ is divisible by }n\}\geq T,\]
or
\item there is some $x\in I$ divisible by all $q\in\mathcal{D}_I(A;L)$.
\end{enumerate}
\end{enumerate}
Then
\[\sum_{h\in\mathfrak{m}_2(A,k,K,T)}C(A;hk)\leq 1/8.\]
\end{lemma}
\begin{proof}
\uses{lem:minor2_ind_bound,lem:minor1_bound}
For any $h\in\mathfrak{m}_2$, let $I_h$ be the interval of length $K$ centred at $kh$. Since $h\not\in\mathfrak{m}_1$, condition 2(a) cannot hold, and therefore there is some $x\in I_h$ divisible by all $q\in\mathcal{D}_{I_h}(A;L)$. In particular, $kh$ is distance at most $K/2$ from some multiple of $[\mathcal{D}_{I_h}(A;L)]$. Since $h\in \mathfrak{m}(A,k,K,T)$, we know $h\not\in \mathfrak{M}(A,k,K)$, and hence $kh$ is distance greater than $K/2$ from any multiple of $[A]$. Since $[\mathcal{Q}]=[A]$, it follows that $\mathcal{D}_{I_h}(A;L)\neq \mathcal{Q}$.

Furthermore, for any $\mathcal{D}\subset \mathcal{Q}$, the number of $h\in\mathfrak{m}_2$ with $\mathcal{D}_{I_h}(A;L)=\mathcal{D}$ is at most $K$ times the number of multiples of $[\mathcal{D}]$ in $[1,k[A]+K]$. The number of multiples of $[\mathcal{D}]$ in $[1,k[A]+K]$ is at most $(k[A]/[\mathcal{D}])+K$, and hence the number of $h\in\mathfrak{m}_2$ with $\mathcal{D}_{I_h}(A;L)=\mathcal{D}$ is at most
\[Kk\frac{[A]}{[\mathcal{D}]} +K^2\leq Kk\prod_{q\in\mathcal{Q}\backslash\mathcal{D}}q+K^2\leq kN^{\abs{\mathcal{Q}\backslash \mathcal{D}}+1}+K^2.\]
In particular, when $\mathcal{D}\neq \mathcal{Q}$, the right-hand side is at most $2kN^{\abs{\mathcal{Q}\backslash \mathcal{D}}+1}$, that is, 
\[\sum_{h\in \mathfrak{m}_2}1_{\mathcal{D}_{I_h}(A;L)=\mathcal{D}}\leq 2kN^{\abs{\mathcal{Q}\backslash \mathcal{D}}+1}.\]
(Here we have used the fact that $[A]\leq [D]\prod_{q\in \mathcal{Q}\backslash \mathcal{D}}q$, which is true because every prime power dividing $[A]$ is in $\mathcal{Q}$, and hence either divides $[D]$ or divides $\prod_{q\in \mathcal{Q}\backslash \mathcal{D}}q$.)

Therefore, for any $\mathcal{D}\subset\mathcal{Q}$, using Lemma~\ref{lem:minor2_ind_bound},
\[\sum_{h\in \mathfrak{m}_2}1_{\mathcal{D}_{I_h}(A;L)=\mathcal{D}}C(A;hk)\leq 2kN^{1-3\abs{\mathcal{Q}_A\backslash\mathcal{D}}}.\]
Since $\mathcal{D}\neq \mathcal{Q}$, this is at most $2kN^{-1-\abs{\mathcal{Q}\backslash \mathcal{D}}}$. 

 Therefore (using the trivial estimate $\abs{\mathcal{Q}}\leq N$)
\begin{align*}
\sum_{h\in\mathfrak{m}_2}C(A;h,k)
&\leq
\frac{2k}{N}\sum_{\mathcal{D}\subseteq\mathcal{Q}}N^{-\abs{\mathcal{Q}\backslash \mathcal{D}}}\\
&= \frac{2k}{N}(1+1/N)^{\abs{\mathcal{Q}}}\\
&\leq 8k/N \leq 1/8.
\end{align*}
\end{proof}

\begin{proposition}\label{prop:fourier}
  \leanok
  \lean{circle_method_prop}
There exists a constant $c>0$ such that the following holds. Suppose that $N$ is sufficiently large. Let $K,L,M,T$ be reals such that $T,L>0$ and $8\leq K<M\leq N$, and let $k$ be an integer such that $1\leq k\leq M/64$. Let $A\subset [M,N]$ be a set of integers such that
\begin{enumerate}
\item $R(A)\in [2/k-1/M,2/k)$,
\item $k$ divides $[A]$,
\item if $q\in\mathcal{Q}_A$ then $q\leq c\min\brac{M/k,\frac{LK^2}{N^2(\log N)^2},\frac{TK^2}{N^2\log N}}$,  and
\item for any interval $I$ of length $K$, either
\begin{enumerate}
\item \[\# \{ n\in A : \textrm{no element of }I\textrm{ is divisible by }n\}\geq T,\]
or
\item there is some $x\in I$ divisible by all $q\in\mathcal{D}_I(A; L)$.
\end{enumerate}
\end{enumerate}
There is some $S\subset A$ such that $R(S)=1/k$.
\end{proposition}
\begin{proof}
\uses{lem:minor_lbound,lem:minor2_bound}
If the conclusion fails, there is an immediate contradiction by combining Lemmas~\ref{lem:minor_lbound}, \ref{lem:minor1_bound} and \ref{lem:minor2_bound}. The only hypothesis that is not immediate to check is the upper bound $[A]<2^{\abs{A}-1}$. This follows since $\lvert A\rvert \geq MR(A)\geq \frac{2}{k}M-1\geq M/k$, and by Lemma~\ref{lem:smooth_lcm} we have $[A]\leq e^{O(cM/k)}$, which is at most $2^{\abs{A}/2}$ assuming $c$ is sufficiently small.
\end{proof}
