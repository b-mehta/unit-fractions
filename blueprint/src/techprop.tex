\chapter{Deduction of main technical proposition}
\label{chap:techprop}

\begin{lemma}\label{lem:find_multiples}
  \uses{def:interval_rare_ppowers}
Suppose $N$ is sufficiently large and $N\geq M\geq N^{1/2}$, and suppose that $A\subset [M,N]$ is a set of integers such that
\[\tfrac{99}{100}\log\log N\leq \omega(n)\leq  2\log\log N\quad\textrm{for all}\quad n\in A.\]
Then for any interval of length $\leq MN^{-2/(\log \log N)}$, if $A_I$ is the set of those $n\in A$ which divide some element of $I$, for all $q\in\mathcal{Q}_A$ such that $R(A_I;q)> 1/2(\log N)^{1/100}$ there exists some integer $x_q\in I$ such that $q\mid x_q$ and
\[\sum_{\substack{r\mid x_q\\ r\in \mathcal{Q}_A}}\frac{1}{r}\geq 0.35\log\log N.\]
\end{lemma}
\begin{proof}
  \uses{lem:usingq,lem:rtop}
Let $I$ be any interval of length $\leq MN^{-2/(\log\log N)}$, and let $q\in\mathcal{Q}_A$ such that $R(A_I;q)> 1/2(\log N)^{1/100}$.

Apply Lemma~\ref{lem:usingq} with $A$ replaced by $A_I$, and $k=\frac{\log\log N}{\log 2+\log\log\log N}$, so that $(\log N)^{1/k}=2\log\log N$. We choose $N$ large enough such that $k\leq c\log\log N$ (where $c$ is the constant in the statement of Lemma~\ref{lem:usingq}) and such that $\log k\geq 100$.

This produces some $d_q$ such that $qd_q>\abs{I}$ and $\omega(d_q)<\tfrac{1}{500}\log\log N$, and
  \[\sum_{\substack{n\in A_I\\ qd_q\mid n\\ (qd_q,n/qd_q)=1}}\frac{qd_q}{n}\gg \frac{1}{(\log N)^{1/100}(\log\log N)^2}\geq \frac{1}{(\log N)^{1/99}},\]
assuming $N$ is sufficiently large. Let
  \[A_I^{(q)}=\{ n/qd_q : n\in A_I\textrm{ with }qd_q\mid n\textrm{ and }(qd_q,n/qd_q)=1\}\]
  so that, the above inequality states that $R(A_I^{(q)})\geq (\log N)^{-1/99}$.

We now claim that $2\log\log N\geq \omega(m) \geq \tfrac{97}{99}\log\log N$ for all $m\in A_I^{(q)}$. This follows from the trivial $\omega(md_q)\geq \omega(m)\geq \omega(md_q)-\omega(d_q)$, and that $2\log\log N\geq \omega(md_q)\geq  \frac{99}{100}\log\log N$ (since $md_q\in A$) and $\omega(d_q)<\frac{1}{500}\log\log N$.

We now apply Lemma~\ref{lem:rtop} with $\epsilon=2/99$. This implies that
  \[\sum_{r\in \mathcal{Q}_{A_I^{(q)}}}\frac{1}{r}\geq \frac{95}{99}e^{-1}\log\log N\geq 0.35\log\log N.\]


    By definition of $A_I$, every $n\in A_I$ with $qd_q\mid n$ must divide some $x\in I$, say $x(q;n)$. We claim that $x(q,n)=x(q,m)$ for all $n,m\in A_I$ with $qd_q\mid n$ and $qd_q\mid m$. Otherwise $qd_q$ divides $\abs{x(q,n)-x(q,m)}$, so $\abs{I}<qd_q\leq \abs{x(q,n)-x(q,m)}$, which is an integer in $[1,\abs{I})$, a contradiction.

  Let $x_q\in I$ be such that $x(q,n)=x_q$ for all $n\in A_I$ with $qd_q \mid n$. We claim that $\mathcal{Q}_{A_I^{(q)}}\subset \{ r\in \mathcal{Q}_A : r\mid x_q\}$, which concludes the proof using the above bound.

If $r\in \mathcal{Q}_{A_I^{(q)}}$ then there exists some $m\in A_I^{(q)}$ such that $r\mid m$ and $(r,m/r)=1$. Let $m=n/qd_q$ where $n\in A$ with $qd_q\mid n$ and $(qd_q,n/qd_q)=(qd_q,m)=1$. Certainly $r\mid n = mqd_q$, since $r\mid m$. Furthermore, $r$ is coprime to $qd_q$ since $r$ divides $m$. Therefore $(r,n/r)=(r,(m/r)qd_q)=(r,m/r)=1$, and hence $r\in \mathcal{Q}_A$.

Finally, we need to show that $r\mid x_q$. Let $m\in A_I^{(q)}$ be such that $r\mid m$. Let $n\in A_I$ be such that $qd_q\mid n$ and $m=n/qd_q$. Then $r\mid n=mqd_q$, and hence since $n\mid x_q$ we have $r\mid x_q$ as required.
\end{proof}

\begin{lemma}\label{lem:good_d}
\leanok
\lean{good_d}
  \uses{def:interval_rare_ppowers,def:rec_sum_local}
 Suppose $N$ is an integer and $\delta,M\in \mathbb{R}$. Suppose that $A\subset [M,N]$ is a set of integers such that, for all $q\in \mathcal{Q}_A$,
\[R(A;q) \geq 2\delta,\]
Then for any finite set of integers $I$, for all $q\in\mathcal{D}_I(A;\delta M)$, if $A_I$ is the set of those $n\in A$ that divide some element of $I$,
\[R(A_I;q)> \delta.\]
\end{lemma}
\begin{proof}
  \leanok
For every $q\in \mathcal{D}_I(A;\delta M$, by definition,
\[\#\{ n\in A_q : n\not\in A_I\} < \frac{\delta M}{q}.\]
Therefore
\[R(A_I;q) = \sum_{\substack{n \in A_q\\n\in A_I}}\frac{q}{n}> \sum_{n \in A_q}\frac{q}{n}-\frac{q}{M}\brac{\frac{\delta M}{q}}.\]
  \[=R(A;q) - \delta \geq \delta.\]
\end{proof}


\begin{proposition}\label{prop:tech_iterative}
  \leanok
  \lean{force_good_properties}
Suppose $N$ is sufficiently large and $N\geq M\geq N^{1/2}$, and suppose that $A\subset [M,N]$ is a set of integers such that
\[\tfrac{99}{100}\log\log N\leq \omega(n)\leq  2\log\log N\quad\textrm{for all}\quad n\in A,\]
\[R(A)\geq (\log N)^{-1/101}\]
and, for all $q\in \mathcal{Q}_A$,
\[R(A;q) \geq (\log N)^{-1/100}.\]
Then either
\begin{enumerate}
\item there is some $B\subset A$ such that $R(B)\geq \tfrac{1}{3}R(A)$ and
\[\sum_{q\in \mathcal{Q}_{B}}\frac{1}{q}\leq \frac{2}{3}\log\log N,\]
or
\item for any interval of length $\leq MN^{-2/(\log \log N)}$, either
\begin{enumerate}
\item \[\# \{ n\in A : \textrm{no element of }I\textrm{ is divisible by }n\}\geq M/\log N,\]
or
\item there is some $x\in I$ divisible by all $q\in\mathcal{D}_I(A;M/2(\log N)^{1/100})$.
\end{enumerate}
\end{enumerate}
\end{proposition}
\begin{proof}
  \uses{lem:find_multiples,lem:good_d,lem:basic,lem:divisor_bound}
Let $I$ be any interval of length $\leq MN^{-2/(\log\log N)}$, and let $A_I$ be those $n\in A$ that divide some element of $I$, and $\mathcal{D}=\mathcal{D}_I(A;M/2(\log N)^{1/00})$. We may assume that $\abs{A\backslash A_I}< M/\log N$.

Let $\mathcal{E}$ be the set of those $q\in\mathcal{Q}_A$ such that $R(A_I;q)> 1/2(\log N)^{1/100}$, so that by Lemma~\ref{lem:good_d} we have $\mathcal{D}\subset \mathcal{E}$. By Lemma~\ref{lem:find_multiples} for each $q\in\mathcal{E}$ there exists $x_q\in I$ such that $q\mid x_q$ and
\[\sum_{\substack{r\mid x_q\\ r\in \mathcal{Q}_A}}\frac{1}{r}\geq 0.35\log\log N.\]


Let $X=\{x_q : q\in \mathcal{E}\}$. Suppose that $\abs{X}\geq 3$, and say $x,y,z\in X$ are three distinct elements. Then
\[\sum_{q\in\mathcal{Q}_A}\frac{1}{q}\geq \sum_{\substack{q\mid x\\ q\in \mathcal{Q}_A}}\frac{1}{q}+\sum_{\substack{q\mid y\\ q\in \mathcal{Q}_A}}\frac{1}{q}+\sum_{\substack{q\mid z\\ q\in \mathcal{Q}_A}}\frac{1}{q}-\sum_{q\mid (x,y)}\frac{1}{q}-\sum_{q\mid (y,z)}\frac{1}{q}-\sum_{q\mid (x,z)}\frac{1}{q},\]
where the last three sums are restricted to prime powers.

Since $\sum_{q\in\mathcal{Q}_A}\frac{1}{q}\leq (1+o(1))\log\log N\leq 1.01\log\log N$, say, by Lemma~\ref{lem:mertens1}, it follows that
\[\sum_{q\mid (x,y)}\frac{1}{q}+\sum_{q\mid (y,z)}\frac{1}{q}+\sum_{q\mid (x,z)}\frac{1}{q}\geq 0.04\log\log N.\]

But since $x,y\in I$ we have, by Lemma~\ref{lem:basic},
  \[\sum_{q\mid (x,y)}\frac{1}{q}\ll \log\log\log N,\]
and similarly for the other sums, and so the left-hand side is $O(\log\log\log N)$, which is a contradiction for large enough $N$.

Therefore $\abs{X}\leq 2$. If $\mathcal{D}$ is empty the second conclusion trivially holds, and therefore we may assume there is some $q\in \mathcal{E}$ and hence some $x_q\in X$, so $\abs{X}\geq 1$. If $\abs{X}=1$ then we are in the second conclusion, letting $x$ be the unique element of $X$, since by construction $x=x_q$ for all $q\in \mathcal{D}$, and hence in particular $q\mid x$ for all $q\in \mathcal{D}_I$ as required.

Finally, we deal with the case $\abs{X}=2$. Say $X=\{w_1,w_2\}$ with $w_1\neq w_2$ (and both are elements of $I$). Let $A^{(i)}=\{n\in A: n\mid w_i\}$ and $A^{(0)}=A\backslash (A^{(1)}\cup A^{(2)})$. Without loss of generality, label the $w_i$ such that $R(A^{(1)})\geq R(A^{(2)}$.  Suppose first that $R(A^{(0)})<R(A)/3$. In this case, since $R(A^{(0)})+R(A^{(1)})+R(A^{(2)})\geq R(A)$, we have $R(A^{(1)}\geq R(A)/3$. Furthermore, note that
\[\sum_{q\mid w_1}\frac{1}{q}\leq \sum_{q\leq N}\frac{1}{q}-\sum_{q\mid w_2}\frac{1}{q}+\sum_{q\mid (w_1,w_2)}\frac{1}{q}.\]
As above, by Lemma~\ref{lem:mertens1}, the assumed lower bound on the sum over $q\mid w_2$, and Lemma~\ref{lem:basic}, for sufficiently large $N$ the right-hand side is
\[\leq \brac{1.01-0.35+0.01}\log\log N\leq \tfrac{2}{3}\log\log N.\]
Every $q\mid \mathcal{Q}_{A^{(1)}}$ divides $w_1$, and hence
\[\sum_{q\in \mathcal{Q}_{A^{(1)}}}\frac{1}{q}\leq\frac{2}{3}\log\log N,\]
and we are in case (1) with $B=A^{(1)}$.

Finally, suppose that $R(A^{(0)})\geq R(A)/3$. We will derive a contradiction, assuming $N$ is large enough.  Let $A'\subset A^{(0)}$ be the set of those $n\in A_I\cap A^{(0)}$ such that if $n\in A_q$ then $q\in\mathcal{E}$. By definition of $\mathcal{E}$ and Mertens' estimate \ref{lem:mertens1},
  \[R(A^{(0)}\backslash A')\leq \frac{\abs{A\backslash A_I}}{M}+\sum_{q\in\mathcal{Q}_A\backslash \mathcal{E}_I}\frac{1}{q}R(A_I;q)\ll \frac{\log\log N}{(\log N)^{1/100}},\]
  and so  in particular, since $R(A)\geq (\log N)^{-1/101}$, we have $R(A')\gg (\log N)^{-1/101}$.

  In particular, $\abs{A'}\gg M/(\log N)^{-1/101}$. Every $n\in A'$ divides some $x\in I$, and there are at most $MN^{-2/\log\log N}$ many $x\in I$, so by the pigeonhole principle there must exist some $x\in I$ (necessarily $x\neq w_1$ and $x\neq w_2$ since $A'\subset A^{(0)}$) such that, if $A''=\{ n\in A' : n\mid x\}$, then
  \[\abs{A''}\gg N^{2/\log\log N}(\log N)^{-1/101},\]
  and hence $\abs{A''}\geq N^{3/2\log\log N}$, say.

  However, if $n\in A''$ then both $n\mid x$ and $n\mid w_1w_2$ (since every $q$ with $n\in A_q$ is in $\mathcal{E}_I$ and so divides either $w_1$ or $w_2$), and hence $n$ divides
  \[(x,w_1w_2)\leq (x,w_1)(x,w_2)\leq \abs{x-w_1}\abs{x-w_2}\leq N^2.\]
  Therefore the size of $A''$ is at most the number of divisors of some fixed integer $m\leq N^2$, which is at most $N^{(1+o(1))2\log 2/\log\log N}$ by Lemma~\ref{lem:divisor_bound}, and hence we have a contradiction for large enough $N$, since $2\log 2< 3/2$.
\end{proof}

\begin{proposition}\label{prop:tech_iterative2}
Suppose $N$ is sufficiently large and $N\geq M\geq N^{1/2}$, and suppose that $A\subset [M,N]$ is a set of integers such that
\[\tfrac{99}{100}\log\log N\leq \omega(n)\leq  2\log\log N\quad\textrm{for all}\quad n\in A,\]
and, for all $q\in \mathcal{Q}_A$,
\[R(A;q) \geq (\log N)^{-1/100},\]
and $\sum_{q\in\mathcal{Q}_A}\frac{1}{q}\leq \frac{2}{3}\log\log N$.
Then for any interval of length $\leq MN^{-2/(\log \log N)}$, either
\begin{enumerate}
\item \[\# \{ n\in A : \textrm{no element of }I\textrm{ is divisible by }n\}\geq M/\log N,\]
or
\item there is some $x\in I$ divisible by all $q\in\mathcal{D}_I(A;M/2(\log N)^{1/100})$.
\end{enumerate}
\end{proposition}
\begin{proof}
  \uses{lem:find_multiples,lem:good_d,lem:basic}
  Let $I$ be any interval of length $\leq MN^{-2/(\log\log N)}$, and let $A_I$ be those $n\in A$ that divide some element of $I$, and $\mathcal{D}=\mathcal{D}_I(A;M/2q(\log N)^{1/00})$. We may assume that $\abs{A\backslash A_I}< M/\log N$.

  By Lemmas~\ref{lem:find_multiples} and \ref{lem:good_d} for each $q\in\mathcal{D}$ there exists $x_q\in I$ such that $q\mid x_q$ and
\[\sum_{\substack{r\mid x_q\\ r\in \mathcal{Q}_A}}\frac{1}{r}\geq 0.35\log\log N.\]



Let $X=\{x_q : q\in \mathcal{D}\}$. Suppose that $\abs{X}\geq 2$, and say $x,y\in X$ are two distinct elements. Then
\[\sum_{q\in\mathcal{Q}_A}\frac{1}{q}\geq \sum_{\substack{q\mid x\\ q\in \mathcal{Q}_A}}\frac{1}{q}+\sum_{\substack{q\mid y\\ q\in \mathcal{Q}_A}}\frac{1}{q}-\sum_{q\mid (x,y)}\frac{1}{q},\]
where the third sum is restricted to prime powers. Using our assumed bounds, this implies
\[\sum_{q\mid (x,y)}\frac{1}{q}\geq \brac{0.7-\frac{2}{3}}\log\log N\geq \frac{1}{100}\log\log N,\]
say.

But since $x,y\in I$ we have, by Lemma~\ref{lem:basic},
  \[\sum_{q\mid (x,y)}\frac{1}{q}\ll \log\log\log N,\]
  which is a contradiction for large enough $N$.

Therefore $\abs{X}\leq 1$. If $\mathcal{D}$ is empty then the second conclusion trivially holds, for any $x\in I$. Therefore there exists some $q\in \mathcal{D}$, therefore there exists some $x_q\in X$, and hence $\abs{X}=1$. Let $x\in I$ be the unique element of $X$. By construction $x=x_q$ for all $q\in \mathcal{D}$, and hence in particular $q\mid x$ for all $q\in \mathcal{D}_I$ as required.
  \end{proof}

\begin{lemma}\label{lem:techmainprec}
Let $N$ be sufficiently large. Let $\epsilon,y,w,M$ be reals such that $M<N$ and $2\leq \lceil y\rceil \leq \lfloor w\rfloor$ and $1/M< \epsilon\log\log N$ and $\frac{2}{w^2}\geq 3\epsilon\log\log N$ and let $A\subset [M,N]$ be such that
\begin{enumerate}
\item $R(A)\geq 2/y+2\epsilon\log\log N$,
\item every $n\in A$ is divisible by some $d\in [y,w]$, 
\item if $q\in \mathcal{Q}_A$ then $q\leq \epsilon M$.
\end{enumerate}
Then there exists some $\emptyset\neq A'\subseteq A$ and $d\in [y,w]$ such that $R(A')\in [2/d-1/M,2/d)$, for all $q\in\mathcal{Q}_{A'}$ we have $R(A';q)>\epsilon$, and there is a multiple of $d$ in $A'$, and there are no multiples of any $m\in [y,d)$ in $A$. 
\end{lemma}
\begin{proof}
\uses{lem:pisq}
In the proof below, we write $t_i=\min(\lceil y\rceil +i, \lfloor w\rfloor)$ for all integer $i\geq 0$ for simplicity. It is convenient to note at the outset that $t_0=\lceil y\rceil$, $t_i\leq w$ for all $i$, for all $i\geq 0$ we have $t_{i+1}\in [t_i,t_i+1]$, and if $t_i< \lfloor w\rfloor$ then $t_{i+1}=t_i+1$. Furthermore, if $i\geq \lceil y\rceil-\lfloor w\rfloor $, then $t_i=\lfloor w\rfloor$. 

We will prove by induction on $i$ that, for all sufficiently large $N$, for all $\epsilon>0$ and reals $y,w,M$ with $2\leq y+1\leq w$ and $1/M< \epsilon\log\log N$ and $\frac{2}{w^2}\geq 3\epsilon\log\log N$, if $A\subset [M,N]$ is such that every $n\in A$ is divisible by some $d\in[y,w]$ and if $q\in \mathcal{Q}_A$ then $q\leq \epsilon M$, then for all $0\leq i$ there exists $ A_i\subseteq A$ and integer $d_i$ with $y\leq d_i\leq t_i$ such that
\begin{enumerate}
\item $R(A_i) \in [2/d_i-1/M, 2/d_i)$, 
\item for all $q\in \mathcal{Q}_{A_i}$ we have $R(A_i;q)> \epsilon$, 
\item there are no multiples of any $m\in [y,d_i)$ in $A$, and
\item either there exists a multiple of $d_i$ in $A_i$, or nothing in $A_i$ is divisible by any $d\in [y,t_i]$. 
\end{enumerate}

Given this inductive claim, the lemma follows by letting $A' = A_{ \lceil y\rceil-\lfloor w\rfloor }$ and $d=d_{ \lceil y\rceil-\lfloor w\rfloor }$, so that then $t_i=\lfloor w\rfloor$, noting that the assumptions imply that $2/d-1/M\geq 2/w-1/M>0$ and hence $R(A')>0$ and so $A'\neq\emptyset$, and furthermore by assumption the second alternative of point (3) cannot hold. It remains to prove the inductive claim.

For the base case $i=0$, we set $d_0=\lceil y\rceil$ and apply Lemma~\ref{lem:pisq} with $\alpha=2/d_0$ to find some $A_0\subseteq A$ such that $R(A_0)\in[2/d_0-1/M,2/d_0)$ and for all $q\in \mathcal{Q}_{A_0}$ we have $\epsilon < R(A_0;q)$. (To see that point (4) holds, note that $d_0$ is the unique integer in $[y,t_0)$.)

For the inductive step, suppose that $A_i,d_i$ are as given for $i\geq 0$. If there exists a multiple of $d_i$ in $A_i$, then we set $d_{i+1}=d_i$ and $A_{i+1}=A_i$. 

If not, then nothing in $A_i$ is divisible by any $d\in [y,t_i]$. We now set $d_{i+1}=t_{i+1}$ and apply Lemma~\ref{lem:pisq} to $A_i$ with $\alpha=2/d_{i+1}$. To be able to apply this, we need to verify that 
\[\alpha=2/d_{i+1}> 4\epsilon\log\log N\]
and
\[R(A_i)\geq \alpha+2\epsilon\log\log N= 2/d_{i+1}+2\epsilon\log\log N.\]
For the first, we note that $d_{i+1}\leq w$, and hence we are done since $2/w>4/w^2\geq  6\epsilon\log\log N$. 

For the second, we note that by assumption, every $n\in A$ is divisible by some $d\in [y,w]$, and hence since we are assuming that nothing in $A_i$ is divisible by any $d\in [y,t_i]$, we must have $t_i<\lfloor w\rfloor$, and hence $d_{i+1}=t_{i+1}=t_i+1\geq d_i+1$. Therefore
\[R(A_i)\geq \frac{2}{d_i}-\frac{1}{M}\geq \frac{2}{d_{i+1}-1}-\frac{1}{M}\geq \frac{2}{d_{i+1}-1}-\epsilon\log\log N,\]
since $1/M \leq \epsilon\log\log N$, and furthermore
\[\frac{2}{d_{i+1}-1}-\frac{2}{d_{i+1}}=\frac{2}{d_{i+1}(d_{i+1}-1)}\geq \frac{2}{d_{i+1}^2}\geq \frac{2}{w^2}\geq 3\epsilon\log\log N,\]
and hence $R(A_i)\geq \alpha+2\epsilon\log\log N$ as required.

Thus we can apply Lemma~\ref{lem:pisq}, to produce some $A_{i+1}\subseteq A_i$ such that $R(A_{i+1})\in [2/d_{i+1}-1/M,2/d_{i+1})$, such that for all $q\in \mathcal{Q}_{A_{i+1}}$ we have $R(A_{i+1};q)>\epsilon$. Furthermore, if there is something in $A_{i+1}$ divisible by some $d\in [y,t_{i+1}]$, then since $A_{i+1}\subseteq A_i$ there must be something in $A_{i+1}$ divisible by some $d\in [y,t_{i+1}]\backslash [y,t_i]$. Since $t_{i+1}\in [t_i,t_i+1]$, the only integer that could be in $[y,t_{i+1}]\backslash [y,t_i]$ is $t_{i+1}=d_{i+1}$, and the proof is complete. 
\end{proof}

\begin{proposition}[Main Technical Proposition]\label{prop:techmain}
\leanok
\lean{technical_prop}
Let $N$ be sufficiently large. Suppose $A\subset [N^{1-1/\log\log N},N]$ and $1\leq y\leq z\leq (\log N)^{1/500}$ are such that
\begin{enumerate}
\item $R(A)\geq 2/y+(\log N)^{-1/200}$,
\item every $n\in A$ is divisible by some $d_1$ and $d_2$ where $y\leq d_1$ and $4d_1\leq d_2\leq z$,
\item every prime power $q$ dividing some $n\in A$ satisfies $q\leq N^{1-6/\log\log N}$, and
\item every $n\in A$ satisfies
\[\tfrac{99}{100}\log\log N\leq \omega(n) \leq 2\log\log N.\]
\end{enumerate}
There is some $S\subset A$ such that $R(S)=1/d$ for some $d\in [y,z]$.
\end{proposition}
\begin{proof}
\uses{prop:tech_iterative,lem:techmainprec,prop:fourier,prop:tech_iterative2}
Let $M=N^{1-1/\log\log N}$. Apply Lemma~\ref{lem:techmainprec} with $w=z/4$ to find some corresponding $A'$ and $d$.  

Suppose first that case (2) of Proposition~\ref{prop:tech_iterative} holds for $A'$. The hypotheses of Proposition~\ref{prop:fourier} are now met with $k=d$, $\eta=1/2(\log N)^{1/100}$, and $K=MN^{-2/\log \log N}$. This yields some $S\subset A'\subset A$ such that $R(S)=1/d_j$ as required.

Otherwise, Proposition~\ref{prop:tech_iterative} yields some $B\subset A'$ such that
\[R(B)\geq 2/3d_j-1/M\geq 1/2d_j+(\log N)^{-1/200}\]
and where $\sum_{q\in\mathcal{Q}_B}\frac{1}{q}\leq \frac{2}{3}\log\log N$. 

We apply Lemma~\ref{lem:techmainprec} again, with $y=4d$ and $w=z$, to produce some corresponding $B'\subset B$ and $d'\in [4d,z]$. Since $\sum_{q\in \mathcal{Q}_{B'}}\frac{1}{q}\leq \sum_{q\in\mathcal{Q}_B}\frac{1}{q}\leq \frac{2}{3}\log\log N$ we can apply Proposition~\ref{prop:tech_iterative2}. The hypotheses of Proposition~\ref{prop:fourier} are now met with $k=d'$ and $\eta,K$ as above, and thus there is some $S\subset B'\subset A$ such that $R(S)=1/d'$ as required. 
\end{proof}
