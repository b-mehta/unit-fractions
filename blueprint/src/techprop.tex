\chapter{Deduction of main technical proposition}
\label{chap:techprop}

\begin{proposition}\label{prop:tech_iterative}
  \leanok
  \lean{force_good_properties}
Suppose $N$ is sufficiently large and $N\geq M\geq N^{1/2}$, and suppose that $A\subset [M,N]$ is a set of integers such that
\[\tfrac{99}{100}\log\log N\leq \omega(n)\leq  2\log\log N\quad\textrm{for all}\quad n\in A,\]
\[R(A)\geq (\log N)^{-1/101}\]
and, for all $q\in \mathcal{Q}_A$,
\[R(A;q) \geq (\log N)^{-1/100}.\]
Then either
\begin{enumerate}
\item there is some $B\subset A$ such that $R(B)\geq \tfrac{1}{3}R(A)$ and
\[\sum_{q\in \mathcal{Q}_{B}}\frac{1}{q}\leq \frac{2}{3}\log\log N,\]
or
\item for any interval of length $\leq MN^{-2/(\log \log N)}$, either
\begin{enumerate}
\item \[\# \{ n\in A : \textrm{no element of }I\textrm{ is divisible by }n\}\geq M/\log N,\]
or
\item there is some $x\in I$ divisible by all $q\in\mathcal{D}_I(A;M/2q(\log N)^{1/100})$.
\end{enumerate}
\end{enumerate}
If $\sum_{q\in\mathcal{Q}_A}\frac{1}{q}\leq \frac{2}{3}\log\log N$ then case (2) is guaranteed.
\end{proposition}
\begin{proof}
\uses{lem:basic,lem:usingq,lem:rtop,lem:mertens1,lem:divisor_bound}
Let $I$ be any interval of length $\leq MN^{-2/(\log\log N)}$, and let $A_I$ be those $n\in A$ that divide some element of $I$. We may assume that $\abs{A\backslash A_I}< M/\log N$ (or else 2(a) holds), and we need to show that either there is some $x\in I$ divisible by all $q\in\mathcal{D}_I$, or the first case holds.

Let $\mathcal{E}_I$ be the set of those $q\in\mathcal{Q}_A$ such that $R(A_I;q)> 1/2(\log N)^{1/100}$. For every $q\in \mathcal{D}_I$, by definition,
\[R(A_I; q) \geq R(A;q) - \brac{\frac{M}{2q(\log N)^{1/100}}}\frac{q}{M}>\frac{1}{2(\log N)^{1/100}},\]
and hence in particular $\mathcal{D}_I\subset \mathcal{E}_I$.

For any $q\in \mathcal{E}_I$ we may therefore apply Lemma~\ref{lem:usingq} with $A$ replaced by $A_I$, and $k$ chosen such that $(\log N)^{1/k}=2\log\log N$. This produces some $d_q$ such that $qd_q>\abs{I}$ and $\omega(d_q)<\tfrac{1}{500}\log\log N$ (provided $N$ is sufficiently large), and
\[\sum_{\substack{n\in A_I\\ qd_q\mid n\\ (qd_q,n/qd_q)=1}}\frac{qd_q}{n}\gg \frac{1}{(\log N)^{1/100}(\log\log N)^2}.\]
By definition of $A_I$, every $n\in A_I$ with $qd_q\mid n$ must divide some $x\in I$ -- in fact, they must all divide the same $x\in I$ (call this $x_q\in I$, say), since all such $x$ are in particular divisible by $qd_q>\abs{I}$, which can divide at most one element in $I$.

Let
\[A_I^{(q)}=\{ n/qd_q : n\in A_I\textrm{ with }qd_q\mid n\textrm{ and }(qd_q,n/qd_q)=1\}\]
so that, assuming $N$ is sufficiently large, $R(A_I^{(q)})\geq (\log N)^{-1/99}$, say. We may therefore apply Lemma~\ref{lem:rtop} with $\epsilon=2/99$ (note that since $\omega(n)\geq \frac{99}{100}\log\log N$ for $n\in A$ and $\omega(d_q)<\frac{1}{500}\log\log N$, we must have $\omega(m)\geq \frac{97}{99}\log\log N$ for all $m\in A_I^{(q)}$). This implies that
\[\sum_{r\in \mathcal{Q}_{A_I^{(q)}}}\frac{1}{r}\geq \frac{95}{99}e^{-1}\log\log N.\]
Trivially, $\mathcal{Q}_{A_I^{(q)}}\subset \mathcal{Q}_A$, and further by choice of $x_q$, all $r\in \mathcal{Q}_{A_I^{(q)}}$ divide $x_q$, and hence
\[\sum_{\substack{r\mid x_q\\ r\in \mathcal{Q}_A}}\frac{1}{r}\geq \frac{95}{99}e^{-1}\log\log N\geq 0.35\log\log N.\]
For any two $n_1\neq n_2\in I$, we have
\[\sum_{q\mid (n_1,n_2)}\frac{1}{q}\ll \log\log\log N\leq 0.01\log\log N\]
for $N$ sufficiently large, by Lemma~\ref{lem:basic}. It follows that if $\sum_{q\in\mathcal{Q}_A}\frac{1}{q}\leq \frac{2}{3}\log\log N$ then there can be at most one such possible value for $x_q\in I$ as $q$ ranges over $\mathcal{E}_I$, and hence this common shared value of $x_q$ is an $x\in I$ divisible by all $q\in\mathcal{E}_I$, and hence certainly by all $q\in\mathcal{D}_I$, as required.

Furthermore, since $\sum_{q\in\mathcal{Q}_A}\frac{1}{q}\leq (1+o(1))\log\log N\leq 1.01\log\log N$, say, there must always be at most two distinct values of $x_q\in I$ as $q$ ranges over $\mathcal{E}_I$. If there is no $x\in I$ divisible by all $q\in\mathcal{D}_I$, there must be exactly two such values, say $w_1$ and $w_2$.

Let $A^{(i)}=\{n\in A: n\mid w_i\}$ and $A^{(0)}=A\backslash (A^{(1)}\cup A^{(2)})$. Since every $q\in\mathcal{Q}_{A^{(1)}}$ divides $w_1$,
\[\sum_{q\in \mathcal{Q}_{A^{(1)}}}\frac{1}{q}\leq \sum_{q\leq N}\frac{1}{q}- \sum_{q\mid w_2}\frac{1}{q}+ \sum_{q\mid (w_1,w_2)}\frac{1}{q}\leq (1-\tfrac{95}{99}e^{-1}+o(1))\log\log N.\]
For large enough $N$, the right-hand side is $\leq \frac{2}{3}\log\log N$, and similarly for $A^{(2)}$. Since $R(A^{(0)})+R(A^{(1)})+R(A^{(2)})\geq R(A)$, we are in the first case choosing $B=A^{(1)}$ or $B=A^{(2)}$, unless $R(A^{(0)})\geq R(A)/3$. In this latter case we will derive a contradiction.

Let $A'\subset A^{(0)}$ be the set of those $n\in A_I\cap A^{(0)}$ such that if $n\in A_q$ then $q\in\mathcal{E}_I$. By definition of $\mathcal{E}_I$ and Mertens' estimate \ref{lemma:mertens1},
\[R(A^{(0)}\backslash A')\leq \frac{\abs{A\backslash A_I}}{M}+\sum_{q\in\mathcal{Q}_A\backslash \mathcal{E}_I}\frac{1}{q}R(A_I;q)\ll \frac{\log\log N}{(\log N)^{1/100}},\]
and so  in particular, since $R(A)\geq (\log N)^{-1/101}$, we have $R(A')\gg (\log N)^{-1/101}$.

In particular, $\abs{A'}\gg M/(\log N)^{-1/101}$. Therefore there must exist some $x\in I$ (necessarily $x\neq w_1$ and $x\neq w_2$ since $A'\subset A^{(0)}$) such that, if $A''=\{ n\in A' : n\mid x\}$, then
\[\abs{A''}\gg N^{2/\log\log N}(\log N)^{-1/101},\]
and hence $\abs{A''}\geq N^{3/2\log\log N}$, say.

However, if $n\in A''$ then both $n\mid x$ and $n\mid w_1w_2$ (since every $q$ with $n\in A_q$ is in $\mathcal{E}_I$ and so divides either $w_1$ or $w_2$), and hence $n$ divides
\[(x,w_1w_2)\leq (x,w_1)(x,w_2)\leq \abs{x-w_1}\abs{x-w_2}\leq N^2.\]
Therefore the size of $A''$ is at most the number of divisors of some fixed integer $m\leq N^2$, which is at most $N^{(1+o(1))2\log 2/\log\log N}$, and hence we have a contradiction for large enough $N$, since $2\log 2< 3/2$.
\end{proof}


\begin{proposition}[Main Technical Proposition]\label{prop:techmain}
\leanok
\lean{technical_prop}
Let $N$ be sufficiently large. Suppose $A\subset [N^{1-1/\log\log N},N]$ and $1\leq y\leq z\leq (\log N)^{1/500}$ are such that
\begin{enumerate}
\item $R(A)\geq 2/y+(\log N)^{-1/200}$,
\item every $n\in A$ is divisible by some $d_1$ and $d_2$ where $y\leq d_1$ and $4d_1\leq d_2\leq z$,
\item every prime power $q$ dividing some $n\in A$ satisfies $q\leq N^{1-6/\log\log N}$, and
\item every $n\in A$ satisfies
\[\tfrac{99}{100}\log\log N\leq \omega(n) \leq 2\log\log N.\]
\end{enumerate}
There is some $S\subset A$ such that $R(S)=1/d$ for some $d\in [y,z]$.
\end{proposition}
\begin{proof}
\uses{prop:tech_iterative,lem:pisq,prop:fourier}
Let $M=N^{1-1/\log\log N}$ and $d_i = \lceil y \rceil +i$. By repeated applications of Lemma~\ref{lem:pisq} we can find a sequence $A\supset A_0\supset A_1\supset\cdots \supset A_{t}$, where $d_t=\lceil z/4\rceil-1$, such that
\[R(A_i)\in [2/d_i-1/M,2/d_i)\quad\textrm{ and }\quad R(A_i;q)\geq (\log N)^{-1/100}\textrm{ for all }q\in \mathcal{Q}_{A_i}.\]
(Note that the hypotheses of Lemma~\ref{lem:pisq} continue to hold since
\[\frac{2}{d_i}-\frac{1}{M}\geq \frac{2}{d_{i}+1}+\frac{1}{(\log N)^{1/200}}\geq \frac{3}{(\log N)^{1/200}}\]
for all $0\leq i\leq t$.) Let $0\leq j\leq t$ be minimal such that there is a multiple of $d_j$ in $A_j$. Such a $j$ exists by assumption, since every $n\in A$ is divisible by some $d\in[y,z/4)$.

Suppose first that case (2) of Proposition~\ref{prop:tech_iterative} holds for $A_j$. The hypotheses of Proposition~\ref{prop:fourier} are now met with $k=d_j$, $\eta=1/2(\log N)^{1/100}$, and $K=MN^{-2/\log \log N}$. This yields some $S\subset A'\subset A$ such that $R(S)=1/d_j$ as required.

Otherwise, Proposition~\ref{prop:tech_iterative} yields some $B\subset A_j$ such that
\[R(B)\geq 2/3d_j-1/M\geq 1/2d_j+(\log N)^{-1/200}\]
and where $\sum_{q\in\mathcal{Q}_B}\frac{1}{q}\leq \frac{2}{3}\log\log N$. Let $e_i = 4d_j+i$ and, once again, repeatedly apply Lemma~\ref{lem:pisq} to find a sequence $B\supset B_0\supset \cdots\supset B_r$, where $e_r=\lfloor z\rfloor$, such that
\[R(B_i)\in [2/e_i-1/M,2/e_i)\quad\textrm{ and }\quad R(B_i;q)\geq (\log N)^{-1/100}\textrm{ for all }q\in \mathcal{Q}_{B_i}.\]
By minimality of $j$, no $d\in [y,d_j)$ divides any element of $A_j$, and hence every $n\in A_j$ is divisible by some $e\in [4d_j,z]$. In particular, there must exist some $0\leq s\leq r$ such that $B_s$ contains a multiple of $e_s$. Furthermore, since $\sum_{q\in \mathcal{Q}_{B_s}}\frac{1}{q}\leq \sum_{q\in\mathcal{Q}_B}\frac{1}{q}\leq \frac{2}{3}\log\log N$ we must be in the second case of Proposition~\ref{prop:tech_iterative}. The hypotheses of Proposition~\ref{prop:fourier} are now met with $k=e_s$ and $\eta,K$ as above, and thus there is some $S\subset B_j\subset A$ such that $R(S)=1/e_s$.
\end{proof}












