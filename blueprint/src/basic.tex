\chapter{Basic Estimates}
\label{chap:basic}

This section contains standard estimates from analytic number theory that will be required.

\begin{lemma}\label{lem:omegasum}\uses{lem:mertensprimes}
For any $X\geq 3$
\[\sum_{n\leq X}\omega(n) = X\log\log X+O(X).\]
\end{lemma}
\begin{proof}
Since $\omega(n) = \sum_{p\leq n}1_{p\mid n}$, the left-hand side equals, after a change in the order of summation,
\[\sum_{p\leq X}\sum_{n\leq X}1_{p\mid n} = \sum_{p\leq X}\left\lfloor\frac{X}{p}\right\rfloor.\]
Since $\lfloor x\rfloor=x+O(1)$, this is equal to
\[X\sum_{p\leq X}\frac{1}{p} +O(\pi(X)) = X\log\log X+O(X),\]
using Lemma~\ref{lem:mertensprimes} and the trivial estimate $\pi(X)\ll X$.
\end{proof}

\begin{lemma}\label{lem:omegasquaredsum}\uses{lem:mertensprimes}
For any $X\geq 3$
\[\sum_{n\leq X}\omega(n)^2 \leq X(\log\log X)^2+O(X\log\log X).\]
\end{lemma}
\begin{proof}
Since $\omega(n) = \sum_{p\leq n}1_{p\mid n}$, the left-hand side is equal to, after expanding the sum and rearranging,
\[\sum_{p,q\leq X}\sum_{n\leq X}1_{p\mid n}1_{q\mid n}.\]
The part of the sum where $p=q$ is
\[\sum_{p\leq X}\sum_{n\leq X}\lfloor X/p\rfloor = X\log\log X+O(X),\]
as in the proof of Lemma~\ref{lem:omegasum}. If $p\neq q$, then $p\mid n$ and $q\mid n$ if and only if $pq\mid n$, and so the sum over $n$ is bounded above by
\[\sum_{n\leq X}1_{pq\mid n}= \lfloor X/pq\rfloor \leq X/pq.\]
Therefore
\[\sum_{\substack{p,q\leq X\\ p\neq q}}\sum_{n\leq X}1_{p\mid n}1_{q\mid n}\leq X\sum_{\substack{p,q\leq X\\ p\neq q}}\frac{1}{pq}\leq X\sum_{p,q\leq X}\frac{1}{pq}\leq X\brac{\sum_{p\leq X}\frac{1}{p}}^2.\]
By Lemma~\ref{lem:mertensprimes} this is $X(\log\log X+O(1))^2 = X(\log\log X)^2+O(X\log\log X)$. The lemma follows by combining the estimates on the two parts of the sum.
\end{proof}

\begin{lemma}[Tur\'{a}n's estimate]\label{lem:turan}
\uses{lem:omegasum,lem:omegasquaredsum}
%\lean{true_iff_false}
For any $X\geq 3$
\[\sum_{n\leq X}(\omega(n) - \log\log X)^2 \ll X\log\log X.\]
\end{lemma}
\begin{proof}
The left-hand side equals
\[\sum_{n\leq X}\omega(n)^2 - 2\log\log X\sum_{n\leq X}\omega(n)+\lfloor X\rfloor (\log\log X)^2.\]
The first summand is at most, by Lemma~\ref{lem:omegasquaredsum},
\[X(\log\log X)^2 +O(X\log\log X).\]
The second summand is equal to, by Lemma~\ref{lem:omegasum},
\[-2\log\log X(X\log\log X+O(X)) = -2X(\log\log X)^2 +O(X\log\log X).\]
The third summand is equal to
\[(X+O(1))(\log\log X)^2 = X(\log\log X)^2 + O(X\log\log X).\]
Therefore the main terms cancel, and
\[\sum_{n\leq X}(\omega(n) - \log\log X)^2\leq O(X\log\log X)\]
as required.
\end{proof}

\begin{lemma}[Chebyshevs' estimate]\label{lem:chebyshev}
\leanok
For any $X\geq 3$
\[\pi(X) \ll \frac{X}{\log X}.\]
\end{lemma}
\begin{proof}
\leanok
\end{proof}

\begin{lemma}[Divisor bound]{\label{lem:divisor_bound}}
\leanok
For any $\epsilon$ such that $0<\epsilon\leq 1$, if $n$ is sufficiently large depending on $\epsilon$, then
\[\tau(n) \leq n^{(1+\epsilon)\frac{\log 2}{\log\log n}}.\]
\end{lemma}
\begin{proof}
We first show that, for any real $K\geq 2$,
\[\tau(n) \leq n^{1/K}K^{2^K}.\]
Write $n$ as the product of unique prime powers $n=p_1^{k_1}\cdots p_r^{k_r}$, so that
\[\frac{\tau(n)}{n^{1/K}} = \frac{\prod_{i=1}^r (k_i+1)}{\prod_{i=1}^r p_i^{k_i/K}} = \prod_{i=1}^r \frac{k_i+1}{p_i^{k_i/K}}.\]
If $p_i>2^K$ then
\[\frac{k_i+1}{p_i^{k_i/K}}\leq \frac{k_i+1}{2^{k_i}}\leq 1,\]
since $1+k \leq 2^k$ for all integer $k\geq 0$ by Bernoulli's inequality. Therefore
\[ \prod_{i=1}^r \frac{k_i+1}{p_i^{k_i/K}} \leq \prod_{\substack{1\leq i\leq r\\ p_i<2^K}}\frac{k_i+1}{p_i^{k_i/K}}.\]
If $p_i<2^K$ then, since $p_i\geq 2$,
\[\frac{k_i+1}{p^{k_i/K}}\leq \frac{k_i+1}{2^{k_i/K}}\leq \frac{k_i+1}{k_i/K+1/2},\]
using the fact that $x+1/2\leq 2^x$ for all $x\geq 0$. Since $K\geq 2$ the denominator here is $\geq (1+k_i)/K$, and so $\frac{k_i+1}{p^{k_i/K}}\leq K$. Therefore
\[\frac{\tau(n)}{n^{1/K}}\leq \prod_{\substack{1\leq i\leq r\\ p_i<2^K}} K\leq K^{\pi(2^K)}\leq K^{2^K},\]
as required.

The second part of the proof is to apply the first part with
\[K=\brac{1+\epsilon/2}^{-1}\frac{\log\log n}{\log 2}.\]
(The right-hand side tends to $\infty$ as $n\to\infty$, so for sufficiently large $n$ we have $K\geq 2$ as required.)

Note that $2^K= (\log n)^{\frac{2}{2+\epsilon}}\leq (\log n)^{1-\epsilon}$ (since $1-\epsilon>\frac{2}{2+\epsilon}$, since $\epsilon\leq 1$) and
\[\log K \leq \log\log\log n.\]
Taking logarithms of the inequality in the first part,
\[\log \tau(n) \leq \frac{\log n}{K}+2^K\log K\leq \log n\brac{\frac{1}{K}+\frac{\log\log\log n}{(\log n)^\epsilon}}\]
\[= \log n\frac{\log 2}{\log\log n}\brac{1+\epsilon/2+\frac{(\log\log\log n)(\log\log n)}{(\log 2)(\log n)^{\epsilon}}}.\]
For any fixed $\epsilon>0$, the function $\frac{(\log\log\log n)(\log\log n)}{(\log 2)(\log n)^{\epsilon}}\to 0$ as $n\to \infty$, so for sufficiently large $n$ it is $\leq \epsilon/2$, and hence
\[\log \tau(n) \leq  \log n\frac{\log 2}{\log\log n}(1+\epsilon)\]
as required.

\end{proof}

\begin{lemma}[Mertens' estimate]\label{lem:mertens1}
There exists a constant $c$ such that
\[\sum_{q\leq X}\frac{1}{q} = \log\log X+c+O(1/\log X),\]
where the sum is restricted to prime powers.
\end{lemma}
\begin{proof}

\end{proof}

\begin{lemma}[Mertens' estimate, just for primes]\label{lem:mertensprimes}\uses{lem:mertens1}
There exists a constant $c$ such that
\[\sum_{p\leq X}\frac{1}{p} = \log\log X+c+O(1/\log X),\]
where the sum is restricted to primes.
\end{lemma}

\begin{lemma}[Mertens' product estimate]\label{lem:mertens2}\uses{lem:mertens1}
For any $X\geq 2$,
\[\prod_{p\leq X}\brac{1-\frac{1}{p}}^{-1}\asymp \log X.\]
\end{lemma}
\begin{proof}

\end{proof}

\begin{lemma}[Sieve of Eratosthenes-Legendre]\label{lem:sieve_eratosthenes}
For any $x,y\geq 0$ and $u\geq v\geq 1$
\[\#\{ n\in (x,x+y] : p\mid n\implies p\not\in [u,v]\} = y\prod_{u\leq p\leq v}\brac{1-\frac{1}{p}}+O(2^v).\]
\end{lemma}
\begin{proof}
Let $P = \prod_{u\leq p\leq v}p$. The left-hand side of the estimate in the lemma can be written as
\[\sum_{x\leq n<x+y}1_{(n,P)=1}.\]
Using the identity $\sum_{d\mid m}\mu(d) = 1$ if $m=1$ and $0$ otherwise, the fact that $d\mid (n,P)$ if and only if $d\mid n$ and $d\mid P$, and that $\sum_{n\leq z}1_{d\mid n}=\left\lfloor\frac{z}{d}\right\rfloor$ for any $z\geq 0$,
\[
\sum_{x< n\leq x+y}1_{(n,P)=1}= \sum_{x< n\leq x+y}\sum_{d\mid (n,P)}\mu(d)\]
\[= \sum_{x< n\leq x+y}\sum_{\substack{d\mid n\\ d\mid P}}\mu(d)\]
\[=\sum_{d\mid P}\mu(d)\sum_{x< n\leq x+y}1_{d\mid n}\]
\[=\sum_{d\mid P}\mu(d)\left(\left\lfloor \frac{x+y}{d}\right\rfloor-\left\lfloor\frac{x}{d}\right\rfloor\right).\]

Since $\lfloor X\rfloor = X+O(1)$ for any $X$, we have
\[\left\lfloor \frac{x+y}{d}\right\rfloor-\left\lfloor\frac{x}{d}\right\rfloor=\frac{y}{d}+O(1)\]
 for any $x,y,d$, and hence
\[\sum_{x< n\leq x+y}1_{(n,P)=1}=y\sum_{d\mid P}\frac{\mu(d)}{d}+O(\sum_{d\mid P}1).\]
We have $\sum_{d\mid P}1=\tau(P)=2^k$ where $k$ is the number of primes in $[u,v]$, which is at most $v$, so the error term here is $O(2^v)$. Finally, expanding out the product shows that
\[\sum_{d\mid P}\frac{\mu(d)}{d} = \prod_{p\in[u,v]}\brac{1-\frac{1}{p}}.\]
Inserting this into the above finishes the proof.
\end{proof}
