\chapter{Technical Lemmas}
\label{chap:techlemmas}


\begin{lemma}
  \label{lem:basic}
  \leanok
  If $0<\abs{n_1-n_2}\leq N$ then
  \[\sum_{q\mid (n_1,n_2)}\frac{1}{q}\ll \log\log\log N,\]
  where the summation is restricted to prime powers.
\end{lemma}
\begin{proof}
  \uses{lem:mertens1}
  If $q\mid (n_1,n_2)$ then $q$ divides $\abs{n_1-n_2}$, and hence in particular $q\leq N$. The contribution of all prime powers $p^r$ with $r\geq 2$ is $O(1)$, and hence it suffices to show that $\sum_{p\mid \abs{n_1-n_2}}\frac{1}{p}\ll \log\log\log N$. Any integer $\leq N$ is trivially divisible by $O(\log N)$ many primes. Clearly summing $1/p$ over $O(\log N)$ many primes is maximised summing over the smallest $O(\log N)$ primes. Since there are $\gg (\log N)^{3/2}$ many primes $\leq (\log N)^2$, we have
  \[\sum_{p\mid \abs{n_1-n_2}}\frac{1}{p}\ll \sum_{p\leq (\log N)^2}\frac{1}{p}\ll \log\log\log N\]
  by Mertens' estimate \ref{lem:mertens1}.
\end{proof}

\begin{lemma}\label{lem:rtop}
  \uses{def:rec_sum_local,def:rec_sum}
Let $1/2>\epsilon>0$ and $N$ be sufficiently large, depending on $\epsilon$. If $A$ is a finite set of integers such that $R(A)\geq (\log N)^{-\epsilon/2}$ and $(1-\epsilon)\log\log N\leq \omega(n)\leq  2\log\log N$ for all $n\in A$ then
\[\sum_{q\in\mathcal{Q}_A}\frac{1}{q} \geq (1-2\epsilon)e^{-1}\log\log N.\]
\leanok
\lean{rec_qsum_lower_bound}
\end{lemma}
\begin{proof}
Since, by definition, every integer $n\in A$ can be written uniquely as $q_1\cdots q_t$ for $q_i\in \mathcal{Q}_A$ for some $t\in I = [(1-\epsilon)\log\log N, 2\log\log N]$, we have that, since $t!\geq (t/e)^t$,
\[R(A)\leq  \sum_{t\in I}\frac{\brac{\sum_{q\in \mathcal{Q}_A}\frac{1}{q}}^t}{t!}\leq \sum_{t\in I}\brac{\frac{e}{t}\sum_{q\in \mathcal{Q}_A}\frac{1}{q}}^t.\]
Since $(ex/t)^t$ is decreasing in $t$ for $x<t$, either $\sum_{q\in \mathcal{Q}_A}\frac{1}{q}\geq (1-\epsilon)\log\log N$ (and we are done), or the summand is decreasing in $t$, and hence we have
\[(\log N)^{-\epsilon/2}\leq R(A)\leq 2\log\log N\brac{\frac{\sum_{q\in \mathcal{Q}_A}\frac{1}{q}}{(1-\epsilon)e^{-1}\log\log N}}^{(1-\epsilon)\log\log N}.\]
The claimed bound follows, using the fact that $e^{-\frac{\epsilon}{2(1-\epsilon)}}\geq 1-\epsilon$ for $\epsilon\in (0,1/2)$, choosing $N$ large enough such that $(2\log\log N)^{2/\log\log N}\leq 1+\epsilon^2$, say.

To verify the inequality $e^{-\frac{\epsilon}{2(1-\epsilon)}}\geq 1-\epsilon$ for $\epsilon\in (0,1/2)$, note that $\frac{\epsilon}{2(1-\epsilon)}\leq \epsilon$ for all $\epsilon\in (0,1/2)$, and hence using $1+x\leq e^{x}$ for all $x\geq -1$, we have 
\[e^{-\frac{\epsilon}{2(1-\epsilon)}}\geq e^{-\epsilon}\geq 1-\epsilon\]
as required. 
\end{proof}

\begin{lemma}\label{lem:smoothsum}
There is a constant $C>0$ such that the following holds. Let $0<y<N$ be some reals, and suppose that $D$ is a finite set of integers such that if $d\in D$ and $p^r\| d$ then $y<p^r\leq N$. Then, for any real $k\geq 1$, 
\[\sum_{d\in D}\frac{k^{\omega(d)}}{d}\leq \left(C\frac{\log N}{\log y}\right)^k.\]
\end{lemma}
\begin{proof}
\uses{lem:mertens1}
We can write every $d\in D$ uniquely as a product of prime powers $d=p_1^{r_1}\cdots p_k^{r_k}$, where for each $i$ either $r_i=1$ and $y<p_i\leq N$, or $r_i\geq 2$ and $p_i\leq N$. Therefore the left-hand side is at most
\[\prod_{y<p\leq N}\left(1+\frac{k}{p}\right)\prod_{p\leq N}\left(1+\frac{k}{p^2}+\cdots\right).\]
The second product is, using $1+kx\leq (1+x)^k$,
\[=\prod_{p\leq N}\left(1+\frac{k}{p(p-1)}\right)\leq \left(\prod_{p\leq N}\left(1+\frac{1}{p(p-1)}\right)\right)^k\leq C_1^k\]
for some constant $C_1>0$, since the product converges. Similarly, 
\[\prod_{y<p\leq N}\left(1+\frac{k}{p}\right)\leq \left(\prod_{y<p\leq N}\left(1+\frac{1}{p}\right)\right)^k,\]
and the product here is $\ll (\log N/\log y)$ by Mertens' bound \ref{lem:mertens1}. Combining these yields the required estimate. 
\end{proof}


\begin{lemma}\label{lem:usingq}
  \leanok
  \lean{find_good_d}
  \uses{def:rec_sum_local}
There is a constant $c>0$ such that the following holds. Let $N\geq M\geq N^{1/2}$ be sufficiently large, and suppose that $k$ is a real number such that $1\leq k \leq c\log\log N$. Suppose that $A\subset [M,N]$ is a set of integers such that $\omega(n)\leq (\log N)^{1/k}$ for all $n\in A$.

For all $q$ such that $R(A;q)\geq 1/\log N$ there exists $d$ such that
\begin{enumerate}
\item $qd > M\exp(-(\log N)^{1-1/k})$,
\item $\omega(d)\leq \tfrac{5}{\log k}\log\log N$, and
\item \[\sum_{\substack{n\in A_q\\qd\mid n\\ (qd,n/qd)=1}}\frac{qd}{n}\gg \frac{R(A;q)}{(\log N)^{2/k}}.\]
\end{enumerate}
\end{lemma}
\begin{proof}\uses{lem:smoothsum}
Fix some $q$ with $R(A;q)\geq (\log N)^{-1/2}$. Let $y=\exp((\log N)^{1-2/k})$. Let $D$ be the set of all $d$ such that if $r$ is a prime power such that $r\mid d$ and $(r,d/r)=1$ then $y<r\leq N$, and furthermore 
\[qd\in (M\exp(-(\log N)^{1-1/k}),N].\]
We first claim that every $n\in A_q$ is divisible by some $qd$ with $d\in D$, such that $(qd,n/qd)=1$. Let $Q_n$ be the set of prime powers $r$ dividing $n$ such that $(r,n/r)=1$, so that $n=\prod_{r\in Q_n}r$. Let 
\[Q_n'=\{ r\in Q_n: r\neq q\textrm{ and }r>y\}.\]
We let $d=\prod_{r\in Q_n'}r$. We need to check that $d\in D$ and $qd\mid n$. It is immediate from the definitions that if a prime power $r$ divides $d$ with $(r,d/r)=1$ then $r\in Q_n'$, whence $r>y$. It is also straightforward to check, by comparing prime powers, that $qd\mid n$, and hence $qd\leq n\leq N$. Furthermore, 
\[\frac{n}{qd}=\prod_{r\in Q_n\backslash Q_n'\cup\{q\}}r\leq y^{\abs{Q_n}}=y^{\omega(n)}\leq \exp((\log N)^{1-1/k}),\]
by definition of $y$ and the hypothesis that $\omega(n)\leq (\log N)^{1/k}$. It follows that $d\in D$ as required. 	

If we let $A_q(d)=\{ n \in A_q : qd\mid n\textrm{ and }(qd,n/qd)=1\}$ then we have shown that $A_q\subseteq \bigcup_{d\in D}A_q(d)$. Therefore
\[R(A;q) = \sum_{n\in A_q}\frac{q}{n}\leq \sum_{d\in D}\sum_{n\in A_q(d)}\frac{q}{n}.\]

We will control the contribution from those $d$ with $\omega(d)>\omega_0= \frac{5}{\log k}\log\log N$ with the trivial bound
\[\sum_{A_q(d)}\frac{q}{n} \leq \sum_{\substack{n\leq N\\ qd\mid n}}\frac{q}{n}\leq \sum_{m\leq N}\frac{q}{dqm}\leq 2\frac{\log N}{d},\]
using the harmonic sum bound $\sum_{m\leq N}\frac{1}{m}\leq \log N+1\leq 2\log N$. 
Therefore 
\[
\sum_{\substack{d\in D\\ \omega(d)>\omega_0}}\sum_{n\in A_q(d)}\frac{q}{n}
\leq 2\log N\sum_{\substack{d\in D\\ \omega(d)> \omega_0}} \frac{1}{d}\]
\[\leq 2k^{-\omega_0}\log N\sum_{d\in D} \frac{k^{\omega(d)}}{d}.\]
By \ref{lem:smoothsum} we have
\[\sum_{d\in D}\frac{k^{\omega(d)}}{d}\leq \left(C\frac{\log N}{\log y}\right)^k.\]
and therefore
\[\sum_{\substack{d\in D\\ \omega(d)>\omega_0}}\sum_{n\in A_q(d)}\frac{q}{n}\leq 2k^{-\omega_0}\log N \left(C\frac{\log N}{\log y}\right)^k.\]
Recalling $\omega_0=\frac{5}{\log k}\log\log N$ and  $y=\exp((\log N)^{1-2/k})$ the right-hand side here is at most 
\[2(\log N)^{-5}C^k\leq 1/2\log N\leq \tfrac{1}{2}R(A;q),\]
for sufficiently large $N$, assuming $k\leq c\log\log N$ for sufficiently small $c$.  

It follows that
\[\tfrac{1}{2}R(A;q)\leq \sum_{\substack{d\in D\\ \omega(d)\leq \omega_0}}\sum_{d\in D}\frac{q}{n} .\]
The result follows by averaging, since by \ref{lem:smoothsum}
\[\sum_{d\in D}\frac{1}{d} \ll \frac{\log N}{\log y}\ll (\log N)^{2/k}.\]
\end{proof}

\begin{lemma}\label{lem:pisqa}
  \leanok
  \lean{pruning_lemma_one}
  \uses{def:rec_sum,def:rec_sum_local}
Let $N$ be sufficiently large, and let $\epsilon>0$ and $A\subset[1,N]$. There exists $B\subset A$ such that
\[R(B)\geq  R(A)-2\epsilon\log\log N\]
and $\epsilon < R(B;q)$ for all $q\in \mathcal{Q}_B$.
\end{lemma}
\begin{proof}
\leanok
We will prove the following claim, valid for any finite set $A\subset \mathbb{N}$ with $0\not\in A$ and $\epsilon>0$, via induction: for all $i\geq 0$ there exists $A_i\subseteq A$ and $Q_i\subseteq \mathcal{Q}_A$ such that 
\begin{enumerate}
\item the sets $\mathcal{Q}_{A_i}$ and $Q_i$ are disjoint,
\item $R(A_i)\geq R(A)- \epsilon\sum_{q\in Q_i}\frac{1}{q}$,
\item either $i\leq \abs{A\backslash A_i}$ or, for all $q'\in \mathcal{Q}_{A_i}$, $\epsilon < R(A_i;q')$. 
\end{enumerate} 

Given this claim, the stated lemma follows by choosing $B=A_{\abs{A}+1}$, noting that in the 5th point above the first alternative cannot hold, and furthermore
\[\epsilon \sum_{q\in Q_{\abs{A}+1}}\frac{1}{q}\leq \epsilon \sum_{q\leq N}\frac{1}{q}\leq 2\epsilon \log\log N,\]
by Lemma~\ref{lem:mertens1}, assuming $N$ is sufficiently large. 

To prove the inductive claim, we begin by choosing $Q_0=\emptyset$, and $A_0=A$. All of the properties are trivial to verify.

Suppose the proposition is true for $i\geq 0$, with associated $A_i,Q_i$. We will now prove it for $i+1$. If it is true that, for all $q'\in\mathcal{Q}_{A_i}$, we have $\epsilon <R(A_i;q')$, then we let $A_{i+1}=A_i$ and $Q_{i+1}=Q_i$. 

Otherwise, let $q'\in \mathcal{Q}_{A_i}$ be such that $R(A_i;q')\leq \epsilon$, set $A_{i+1} = A_i\backslash \{ n\in A_i : q'\mid n \textrm{ and }(q',n/q')=1\}$, and $Q_{i+1}=Q_i\cup \{q'\}$.

It remains to verify the required properties. That $Q_{i+1}\subseteq \mathcal{Q}_A$ follows from $Q_i\subseteq \mathcal{Q}_A$ and $q'\in \mathcal{Q}_{A_i}\subseteq \mathcal{Q}_A$  (using the general fact that if $B\subseteq A$ then $\mathcal{Q}_B\subseteq \mathcal{Q}_A$).

For point (1), since $A_{i+1}\subseteq A_i$, and hence $\mathcal{Q}_{A_{i+1}}\subseteq \mathcal{Q}_{A_i}$, it suffices to show that $q'\not\in \mathcal{Q}_{A_{i+1}}$. If this were not true, there would be some $n\in A_{i+1}$ such that $q'\mid n$ and $(q',n/q')=1$, contradiction to the definition of $A_{i+1}$.

For point (2), we note that 
\[R(A_{i+1})= R(A_i)-\sum_{\substack{n\in A_i\\ q'\mid n\\ (q',n/q')=1}}\frac{1}{n} = R(A_i)-\frac{1}{q'}R(A_i;q')\geq R(A_i)-\epsilon \frac{1}{q'},\]
and point (2) follows from induction and the observation that
\[\frac{1}{q'}+\sum_{q\in Q_i}\frac{1}{q}=\sum_{q\in Q_{i+1}}\frac{1}{q},\]
since $q'\in\mathcal{Q}_{A_i}$ and hence $q'\not\in Q_{i}$. 

Finally, for point (3), we note that since we assume that it is not true that for all $q'\in\mathcal{Q}_{A_i}$, we have $\epsilon <R(A_i;q')$, it must be true that $i\leq \abs{A\backslash A_{i}}$. We now claim that $i+1\leq \abs{A\backslash A_{i+1}}$, for which it suffices to show that $A_{i+1}\subsetneq A_i$. This is true since $q'\in\mathcal{Q}_{A_i}$, and hence the set $\{ n\in A_i : q'\mid n \textrm{ and }(q',n/q')=1\}$ is not empty. 
\end{proof}

\begin{lemma}\label{lem:pisq}
  \leanok
  \lean{pruning_lemma_two}
Suppose that $N$ is sufficiently large and $N>M$. Let $\epsilon,\alpha$ be reals such that $\alpha > 4\epsilon\log\log N$ and $A\subset [M,N]$ be a set of integers such that
\[R(A) \geq \alpha+2\epsilon\log\log N\]
and if $q\in\mathcal{Q}_A$ then $q\leq \epsilon M$.

There is a subset $B\subset A$ such that $R(B)\in [\alpha-1/M,\alpha)$ and, for all $q\in \mathcal{Q}_B$, we have $\epsilon < R(B;q)$. 
\end{lemma}
\begin{proof}
\uses{lem:pisqa}
\leanok
We will prove the following, for all $N$ sufficiently large, real $0<M<N$, reals $\epsilon>0$ and $\alpha > 4\epsilon\log\log N$ and finite sets $A\subseteq [M,N]$ such that $R(A)\geq \alpha$ and $R(A;q)> \epsilon$ for all $q\in\mathcal{Q}_A$, by induction: for all $i\geq 0$ there exists $A_i\subseteq A$ such that 
\begin{enumerate}
\item $R(A_i)\geq \alpha-1/M$, 
\item $R(A_i;q) > \epsilon$ for all $q\in\mathcal{Q}_{A_i}$, and
\item either $i\leq \abs{A\backslash A_i}$ or $R(A_i)<\alpha$. 
\end{enumerate}

Given this claim, we prove the lemma as follows. Let $A'\subseteq A$ be as given by Lemma~\ref{lem:pisqa}, so that the hypotheses of the inductive claim are satisfied for $A'$. We now apply the inductive claim, and choose $B=A_{\abs{A}+1}$, noting that the first alternative of point (3) cannot hold. 

It remains to prove the inductive claim. The base case $i=0$ is immediate by hypotheses, choosing $A_0=A$. 

We now fix some $i\geq 0$, with associated $A_i$ as above, and prove the claim for $i+1$. If $R(A_i)<\alpha$ then we set $A_{i+1}=A_i$. Otherwise, we have $R(A_i)\geq \alpha$. By Lemma~\ref{lem:pisqa} there exists $B\subset A_i$ such that
\[R(B)\geq  R(A_i)-4\epsilon\log\log N\geq \alpha - 4\epsilon\log\log N>0\]
and $2\epsilon < R(B;q)$ for all $q\in \mathcal{Q}_B$. In particular, since $R(B)>0$, the set $B$ is not empty. Let $x\in B$ be arbitrary, and set $A_{i+1}=A_i\backslash \{x\}$. 

We note that since $x\in B\subseteq A$, we have $x\geq M$, and hence $R(A_{i+1})=R(A_i)-1/x\geq \alpha-1/M$, so point (1) is satisfied. Furthermore, since $R(A_i)\geq \alpha$, we have $i\leq \abs{A\backslash A_i}$, and hence $i+1 \leq \abs{A\backslash A_{i+1}}$ and point (3) is satisfied. 

For point (2), let $q\in \mathcal{Q}_{A_{i+1}}\subseteq \mathcal{Q}_{A_i}$. If it is not true that $q\mid x$ and $(q,x/q)=1$ then
\[R(A_{i+1};q) = R(A_i; q)>\epsilon.\]
Otherwise, if $q\mid x$ and $(q,x/q)=1$, then $q\in\mathcal{Q}_B$. It follows that, since $B\subseteq A_i$, we have
\[R(A_{i+1};q)\geq R(B;q)-\frac{q}{x}>2\epsilon -\frac{q}{M}\geq \epsilon,\]
since $q\leq \epsilon M$. The inductive claim now follows. 
\end{proof}
